\PassOptionsToPackage{unicode=true}{hyperref} % options for packages loaded elsewhere
\PassOptionsToPackage{hyphens}{url}
%
\documentclass[]{book}
\usepackage{lmodern}
\usepackage{amssymb,amsmath}
\usepackage{ifxetex,ifluatex}
\usepackage{fixltx2e} % provides \textsubscript
\ifnum 0\ifxetex 1\fi\ifluatex 1\fi=0 % if pdftex
  \usepackage[T1]{fontenc}
  \usepackage[utf8]{inputenc}
  \usepackage{textcomp} % provides euro and other symbols
\else % if luatex or xelatex
  \usepackage{unicode-math}
  \defaultfontfeatures{Ligatures=TeX,Scale=MatchLowercase}
\fi
% use upquote if available, for straight quotes in verbatim environments
\IfFileExists{upquote.sty}{\usepackage{upquote}}{}
% use microtype if available
\IfFileExists{microtype.sty}{%
\usepackage[]{microtype}
\UseMicrotypeSet[protrusion]{basicmath} % disable protrusion for tt fonts
}{}
\IfFileExists{parskip.sty}{%
\usepackage{parskip}
}{% else
\setlength{\parindent}{0pt}
\setlength{\parskip}{6pt plus 2pt minus 1pt}
}
\usepackage{hyperref}
\hypersetup{
            pdftitle={The Little CPPler},
            pdfborder={0 0 0},
            breaklinks=true}
\urlstyle{same}  % don't use monospace font for urls
\usepackage[verbose,a4paper,textwidth=190mm,textheight=260mm,voffset=7mm,lmargin=12mm]{geometry}
\usepackage{color}
\usepackage{fancyvrb}
\newcommand{\VerbBar}{|}
\newcommand{\VERB}{\Verb[commandchars=\\\{\}]}
\DefineVerbatimEnvironment{Highlighting}{Verbatim}{commandchars=\\\{\}}
% Add ',fontsize=\small' for more characters per line
\newenvironment{Shaded}{}{}
\newcommand{\AlertTok}[1]{\textcolor[rgb]{1.00,0.00,0.00}{\textbf{#1}}}
\newcommand{\AnnotationTok}[1]{\textcolor[rgb]{0.38,0.63,0.69}{\textbf{\textit{#1}}}}
\newcommand{\AttributeTok}[1]{\textcolor[rgb]{0.49,0.56,0.16}{#1}}
\newcommand{\BaseNTok}[1]{\textcolor[rgb]{0.25,0.63,0.44}{#1}}
\newcommand{\BuiltInTok}[1]{#1}
\newcommand{\CharTok}[1]{\textcolor[rgb]{0.25,0.44,0.63}{#1}}
\newcommand{\CommentTok}[1]{\textcolor[rgb]{0.38,0.63,0.69}{\textit{#1}}}
\newcommand{\CommentVarTok}[1]{\textcolor[rgb]{0.38,0.63,0.69}{\textbf{\textit{#1}}}}
\newcommand{\ConstantTok}[1]{\textcolor[rgb]{0.53,0.00,0.00}{#1}}
\newcommand{\ControlFlowTok}[1]{\textcolor[rgb]{0.00,0.44,0.13}{\textbf{#1}}}
\newcommand{\DataTypeTok}[1]{\textcolor[rgb]{0.56,0.13,0.00}{#1}}
\newcommand{\DecValTok}[1]{\textcolor[rgb]{0.25,0.63,0.44}{#1}}
\newcommand{\DocumentationTok}[1]{\textcolor[rgb]{0.73,0.13,0.13}{\textit{#1}}}
\newcommand{\ErrorTok}[1]{\textcolor[rgb]{1.00,0.00,0.00}{\textbf{#1}}}
\newcommand{\ExtensionTok}[1]{#1}
\newcommand{\FloatTok}[1]{\textcolor[rgb]{0.25,0.63,0.44}{#1}}
\newcommand{\FunctionTok}[1]{\textcolor[rgb]{0.02,0.16,0.49}{#1}}
\newcommand{\ImportTok}[1]{#1}
\newcommand{\InformationTok}[1]{\textcolor[rgb]{0.38,0.63,0.69}{\textbf{\textit{#1}}}}
\newcommand{\KeywordTok}[1]{\textcolor[rgb]{0.00,0.44,0.13}{\textbf{#1}}}
\newcommand{\NormalTok}[1]{#1}
\newcommand{\OperatorTok}[1]{\textcolor[rgb]{0.40,0.40,0.40}{#1}}
\newcommand{\OtherTok}[1]{\textcolor[rgb]{0.00,0.44,0.13}{#1}}
\newcommand{\PreprocessorTok}[1]{\textcolor[rgb]{0.74,0.48,0.00}{#1}}
\newcommand{\RegionMarkerTok}[1]{#1}
\newcommand{\SpecialCharTok}[1]{\textcolor[rgb]{0.25,0.44,0.63}{#1}}
\newcommand{\SpecialStringTok}[1]{\textcolor[rgb]{0.73,0.40,0.53}{#1}}
\newcommand{\StringTok}[1]{\textcolor[rgb]{0.25,0.44,0.63}{#1}}
\newcommand{\VariableTok}[1]{\textcolor[rgb]{0.10,0.09,0.49}{#1}}
\newcommand{\VerbatimStringTok}[1]{\textcolor[rgb]{0.25,0.44,0.63}{#1}}
\newcommand{\WarningTok}[1]{\textcolor[rgb]{0.38,0.63,0.69}{\textbf{\textit{#1}}}}
\setlength{\emergencystretch}{3em}  % prevent overfull lines
\providecommand{\tightlist}{%
  \setlength{\itemsep}{0pt}\setlength{\parskip}{0pt}}
\setcounter{secnumdepth}{0}
% Redefines (sub)paragraphs to behave more like sections
\ifx\paragraph\undefined\else
\let\oldparagraph\paragraph
\renewcommand{\paragraph}[1]{\oldparagraph{#1}\mbox{}}
\fi
\ifx\subparagraph\undefined\else
\let\oldsubparagraph\subparagraph
\renewcommand{\subparagraph}[1]{\oldsubparagraph{#1}\mbox{}}
\fi

% set default figure placement to htbp
\makeatletter
\def\fps@figure{htbp}
\makeatother

\usepackage{framed}
%\renewenvironment{Shaded}{\begin{framed}}{\end{framed}}
%\usepackage{mdframed}
%\renewenvironment{Shaded}{\begin{mdframed}}{\end{mdframed}}

\usepackage{graphicx}

%\usepackage{layout} % package to look at the page layout

%\usepackage{fancyhdr}
%\usepackage{lastpage}
%\pagestyle{fancy}
%\lhead{hi}
%\rhead{ho}

\usepackage[l2tabu, orthodox]{nag} % this is to make sure I use some modern LaTeX tactics

%%%
% Custom footnote taken from https://tex.stackexchange.com/a/78227
\makeatletter
\newcommand*{\myfnsymbolsingle}[1]{%
  \ensuremath{%
    \ifcase#1% 0
    \or % 1
      \dagger
    \or % 2
      \ddagger
    \or % 3
      \mathsection
    \else % >= 4
      \@ctrerr
    \fi
  }%
}
\makeatother

\newcommand*{\myfnsymbol}[1]{%
  \myfnsymbolsingle{\value{#1}}%
}

% remove upper boundary by multiplying the symbols if needed
\usepackage{alphalph}
\newalphalph{\myfnsymbolmult}[mult]{\myfnsymbolsingle}{}

\renewcommand*{\thempfootnote}{%
  \myfnsymbolmult{\value{mpfootnote}}%
}

%\renewcommand{\thefootnote}{\fnsymbol{footnote}}
%%%

\hypersetup{bookmarks=true,
            pdfauthor={},
            colorlinks=true,
            citecolor=blue,
            urlcolor=blue,
            linkcolor=magenta}
\usepackage{fontspec}
\usepackage[textsize=tiny,%
            bordercolor=yellow,%
            color=yellow,
            textwidth=4cm]{todonotes}
\usepackage{setspace} % for pandoc-citeproc-preamble
%\setmainfont{Noto Serif}

\newcommand{\inlinetodo}[1]{\todo[inline,size=\normalsize]{#1}}

\title{The Little CPPler}
\date{}

\begin{document}
\maketitle

\listoftodos

\inlinetodo{idea: at some point in the code explain why is the book written in English and
no any other language}

\inlinetodo{idea: have a couple of exercises that could only be done in groups, and make
students do an activity at the start and middle of the class with totally random people to
solve those activities (talleres)}

\hypertarget{what-to-expect-from-this-book}{%
\chapter{What to expect from this
book}\label{what-to-expect-from-this-book}}

\inlinetodo{A phrase telling them what to do with the book}

This is NOT a reference book, this is a teaching book, it is for anybody
to acquire experience writing code in C++.

Learn like you do, don't learn just to accumulate. Learning is
rewarding, but teach/apply what you learn, from day one.

\hypertarget{prerequisites}{%
\chapter{Prerequisites}\label{prerequisites}}

\hypertarget{tools}{%
\section{Tools}\label{tools}}

\inlinetodo{Tools neccesary to follow the book (for everything to work), note: the source
code can be found in a repo, though}

\hypertarget{using-the-tools}{%
\section{Using the tools}\label{using-the-tools}}

\inlinetodo{Ask to open a file with a text editor}

\inlinetodo{Extend explanation on how to compile}

\inlinetodo{ask to modify the file and see what happens}

Given the file \texttt{000-hello-world.cc}:

\begin{framed}

\begin{Shaded}
\begin{Highlighting}[]
\PreprocessorTok{#include }\ImportTok{<iostream>}

\DataTypeTok{int}\NormalTok{ main()}
\NormalTok{\{}
  \BuiltInTok{std::}\NormalTok{cout << }\StringTok{"Hello World!"}\NormalTok{ << }\BuiltInTok{std::}\NormalTok{endl;}
  \ControlFlowTok{return} \DecValTok{0}\NormalTok{;}
\NormalTok{\}}
\end{Highlighting}
\end{Shaded}

\end{framed}

compile the file into a executable with either \texttt{clang++} or
\texttt{g++}:

\inlinetodo{add std option to make it C++11 compilant}

\begin{framed}

\begin{Shaded}
\begin{Highlighting}[]
\FunctionTok{clang}\NormalTok{++ 000-hello-world.cc -o 000-hello-world.exe}
\ExtensionTok{g++}\NormalTok{ 000-hello-world.cc -o 000-hello-world.exe}
\end{Highlighting}
\end{Shaded}

\end{framed}

and run it with:

\begin{framed}

\begin{Shaded}
\begin{Highlighting}[]
\ExtensionTok{./000-hello-world.exe}
\end{Highlighting}
\end{Shaded}

\end{framed}

the output of the program in terminal will be:

\begin{framed}

\begin{verbatim}
Hello World!
\end{verbatim}

\end{framed}

\hypertarget{how-to-read-this-book}{%
\chapter{How to read this book}\label{how-to-read-this-book}}

Yeah, you should learn first how to read this book! :P

\inlinetodo{Explain how the book is supposed to be read (try to answer question with answer
covered, then reveal answer and try understand what is it doing)}

\newpage

\hypertarget{block-1-basics}{%
\chapter{Block 1: Basics}\label{block-1-basics}}

\hypertarget{getting-to-know-c11}{%
\section{Getting to know C++(11)}\label{getting-to-know-c11}}

\inlinetodo{Add interludes asking whoever is reading to pause for a while to recover from
so much info}

\inlinetodo{Add a space before footnotes}

\vspace{2mm}\noindent\hrulefill{}

\begin{minipage}{\linewidth}\noindent
{\tiny 000.}\\
\begin{minipage}[t]{.485\linewidth}

What do you think the following code will output after compiling and
running it?

\begin{framed}

\begin{Shaded}
\begin{Highlighting}[]
\PreprocessorTok{#include }\ImportTok{<iostream>}

\DataTypeTok{int}\NormalTok{ main()}
\NormalTok{\{}
  \BuiltInTok{std::}\NormalTok{cout << }\StringTok{"Hello World!"}\NormalTok{ << }\BuiltInTok{std::}\NormalTok{endl;}
  \ControlFlowTok{return} \DecValTok{0}\NormalTok{;}
\NormalTok{\}}
\end{Highlighting}
\end{Shaded}

\end{framed}

\end{minipage}
\hfill
\begin{minipage}[t]{.485\linewidth}

The output is:

\begin{framed}

\begin{verbatim}
Hello World!
\end{verbatim}

\end{framed}

\end{minipage}
\end{minipage}

\vspace{2mm}\noindent\hrulefill{}

\begin{minipage}{\linewidth}\noindent
{\tiny 001.}\\
\begin{minipage}[t]{.485\linewidth}

Now, what if you compile this other file:

\begin{framed}

\begin{Shaded}
\begin{Highlighting}[]
\PreprocessorTok{#include }\ImportTok{<iostream>}

\DataTypeTok{int}\NormalTok{ main()}
\NormalTok{\{}
  \BuiltInTok{std::}\NormalTok{cout << }\StringTok{"Hello World!"}\NormalTok{ << }\BuiltInTok{std::}\NormalTok{endl;}
  \BuiltInTok{std::}\NormalTok{cout << }\StringTok{"I'm a program example and I'm "}
\NormalTok{            << }\StringTok{"in English."}
\NormalTok{            << }\BuiltInTok{std::}\NormalTok{endl;}
  \ControlFlowTok{return} \DecValTok{0}\NormalTok{;}
\NormalTok{\}}
\end{Highlighting}
\end{Shaded}

\end{framed}

\end{minipage}
\hfill
\begin{minipage}[t]{.485\linewidth}

The output is:

\begin{framed}

\begin{verbatim}
Hello World!
I'm a program example and I'm in English.
\end{verbatim}

\end{framed}

Pay close attention to the output, there are three sentences surrounded
by quotation marks (\texttt{"}), but there are only two lines in the
output. Why?

\end{minipage}
\end{minipage}

\vspace{2mm}\noindent\hrulefill{}

\begin{minipage}{\linewidth}\noindent
{\tiny 002.}\\
\begin{minipage}[t]{.485\linewidth}

If we change our example slightly (notice the semicolon (\texttt{;}))
what do you think it will happen?

\begin{framed}

\begin{Shaded}
\begin{Highlighting}[]
\PreprocessorTok{#include }\ImportTok{<iostream>}

\DataTypeTok{int}\NormalTok{ main()}
\NormalTok{\{}
  \BuiltInTok{std::}\NormalTok{cout << }\StringTok{"Hello World!"}\NormalTok{ << }\BuiltInTok{std::}\NormalTok{endl;}
  \BuiltInTok{std::}\NormalTok{cout << }\StringTok{"I'm a program example and I'm "}\NormalTok{;}
\NormalTok{            << }\StringTok{"in English."}
\NormalTok{            << }\BuiltInTok{std::}\NormalTok{endl;}
  \ControlFlowTok{return} \DecValTok{0}\NormalTok{;}
\NormalTok{\}}
\end{Highlighting}
\end{Shaded}

\end{framed}

\end{minipage}
\hfill
\begin{minipage}[t]{.485\linewidth}

Well, it doesn't compiles! We get an error similar to:

\begin{framed}

\begin{verbatim}
:7:13: error: expected expression
            << "in English."
            ^
\end{verbatim}

\end{framed}

It is telling us that it was expecting something (a
\VERB|\BuiltInTok{std::}\NormalTok{cout}| for example) before
\VERB|\NormalTok{<<}|.

Try removing or adding random characters (anywhere) to the example and
you will find that the compiler just admits a certain arrangement of
characters and not much more. But, why? Well, the compiler just
understands the grammar of C++ as we just understand the grammar of our
human languages. Going a little further with the analogy, we can
understand the grammar of any human language (its parts (verbs,
prepositions, \ldots{}) and how are they connected) but we can only
understand the meaning (semantics) of those languages we have studied
(or our mother tongues).

\end{minipage}
\end{minipage}

\vspace{2mm}\noindent\hrulefill{}

\begin{minipage}{\linewidth}\noindent
{\tiny 003.}\\
\begin{minipage}[t]{.485\linewidth}

Does the following program compiles. If yes, what is its output?

\begin{framed}

\begin{Shaded}
\begin{Highlighting}[]
\PreprocessorTok{#include }\ImportTok{<iostream>}

\DataTypeTok{int}\NormalTok{ main()}
\NormalTok{\{}
  \BuiltInTok{std::}\NormalTok{cout << }\StringTok{"Hello World!"}\NormalTok{ << }\BuiltInTok{std::}\NormalTok{endl;}
  \BuiltInTok{std::}\NormalTok{cout << }\StringTok{"I'm a program"}
\NormalTok{            << }\BuiltInTok{std::}\NormalTok{endl}
\NormalTok{            << }\StringTok{"example and I'm"}
\NormalTok{            << }\BuiltInTok{std::}\NormalTok{endl}
\NormalTok{            << }\StringTok{"in English."}
\NormalTok{            << }\BuiltInTok{std::}\NormalTok{endl;}
  \ControlFlowTok{return} \DecValTok{0}\NormalTok{;}
\NormalTok{\}}
\end{Highlighting}
\end{Shaded}

\end{framed}

\end{minipage}
\hfill
\begin{minipage}[t]{.485\linewidth}

Yep, it in fact compiles, and its output is:

\begin{framed}

\begin{verbatim}
Hello World!
I'm a program
example and I'm
in English.
\end{verbatim}

\end{framed}

Notice how \VERB|\BuiltInTok{std::}\NormalTok{endl}| puts text in a new
line, that's in fact its whole job.

\end{minipage}
\end{minipage}

\vspace{2mm}\noindent\hrulefill{}

\begin{minipage}{\linewidth}\noindent
{\tiny 004.}\\
\begin{minipage}[t]{.485\linewidth}

Well that's getting boring. What if we try something different for a
change. What is the otput of this program:

\begin{framed}

\begin{Shaded}
\begin{Highlighting}[]
\PreprocessorTok{#include }\ImportTok{<iostream>}

\DataTypeTok{int}\NormalTok{ main()}
\NormalTok{\{}
  \BuiltInTok{std::}\NormalTok{cout << }\StringTok{"Adding two numbers: "}
\NormalTok{            << }\DecValTok{2}\NormalTok{ + }\DecValTok{3}
\NormalTok{            << }\BuiltInTok{std::}\NormalTok{endl;}
  \ControlFlowTok{return} \DecValTok{0}\NormalTok{;}
\NormalTok{\}}
\end{Highlighting}
\end{Shaded}

\end{framed}

\end{minipage}
\hfill
\begin{minipage}[t]{.485\linewidth}

Nice\footnote{this is a footnote, read all of them, they may tell you
  little things that the main text won't.}

\begin{framed}

\begin{verbatim}
Adding two numbers: 5
\end{verbatim}

\end{framed}

\end{minipage}
\end{minipage}

\vspace{2mm}\noindent\hrulefill{}

\begin{minipage}{\linewidth}\noindent
{\tiny 005.}\\
\begin{minipage}[t]{.485\linewidth}

Let's try something a little more complex\footnote{Here you can see only
  a snippet of the whole code. The complete code the snippet represents
  can be found in the source accompaning this book.

  From now on, all code will be given on snippets for simplicity but
  remember that they are that, snippets, uncomplete pieces of code that
  need your help to get complete.}

\begin{framed}

\begin{Shaded}
\begin{Highlighting}[]
\BuiltInTok{std::}\NormalTok{cout}
\NormalTok{  << }\StringTok{"A simple operation between "}\NormalTok{ << }\DecValTok{3}
\NormalTok{  << }\StringTok{" "}\NormalTok{ << }\DecValTok{5}\NormalTok{ << }\StringTok{" "}\NormalTok{ << }\DecValTok{20}\NormalTok{ << }\StringTok{": "}
\NormalTok{  << (}\DecValTok{3+5}\NormalTok{)*}\DecValTok{20}\NormalTok{ << }\BuiltInTok{std::}\NormalTok{endl;}
\end{Highlighting}
\end{Shaded}

\end{framed}

\end{minipage}
\hfill
\begin{minipage}[t]{.485\linewidth}

\begin{framed}

\begin{verbatim}
A simple operation between 3 5 20: 160
\end{verbatim}

\end{framed}

\end{minipage}
\end{minipage}

\vspace{2mm}\noindent\hrulefill{}

\begin{minipage}{\linewidth}\noindent
{\tiny 006.}\\
\begin{minipage}[t]{.485\linewidth}

What is the purpose of
\VERB|\NormalTok{<< }\StringTok{" "}\NormalTok{ <<}| in the code?

\end{minipage}
\hfill
\begin{minipage}[t]{.485\linewidth}

\VERB|\NormalTok{<< }\StringTok{" "}\NormalTok{ <<}| adds an space
between the numbers otherwise the output would look weird.

\end{minipage}
\end{minipage}

\vspace{2mm}\noindent\hrulefill{}

\begin{minipage}{\linewidth}\noindent
{\tiny 007.}\\
\begin{minipage}[t]{.485\linewidth}

What if we remove all spaces from the last example?

\begin{framed}

\begin{Shaded}
\begin{Highlighting}[]
\BuiltInTok{std::}\NormalTok{cout}
\NormalTok{  << }\StringTok{"A simple operation between "}\NormalTok{ << }\DecValTok{3}
\NormalTok{  << }\DecValTok{5}\NormalTok{ << }\DecValTok{20}\NormalTok{ << }\StringTok{": "}
\NormalTok{  << (}\DecValTok{3+5}\NormalTok{)*}\DecValTok{20}\NormalTok{ << }\BuiltInTok{std::}\NormalTok{endl;}
\end{Highlighting}
\end{Shaded}

\end{framed}

\end{minipage}
\hfill
\begin{minipage}[t]{.485\linewidth}

\begin{framed}

\begin{verbatim}
A simple operation between 3520: 160
\end{verbatim}

\end{framed}

It looks aweful, doesn't it? Spaces are important as formatting!

\end{minipage}
\end{minipage}

\vspace{2mm}\noindent\hrulefill{}

\begin{minipage}{\linewidth}\noindent
{\tiny 008.}\\
\begin{minipage}[t]{.485\linewidth}

Let's try it now multiline:

\begin{framed}

\begin{Shaded}
\begin{Highlighting}[]
\BuiltInTok{std::}\NormalTok{cout}
\NormalTok{  << }\StringTok{"A simple operation between "}\NormalTok{ << }\BuiltInTok{std::}\NormalTok{endl}
\NormalTok{  << }\DecValTok{3}\NormalTok{ << }\BuiltInTok{std::}\NormalTok{endl}
\NormalTok{  << }\DecValTok{5}\NormalTok{ << }\BuiltInTok{std::}\NormalTok{endl}
\NormalTok{  << }\DecValTok{20}\NormalTok{ << }\StringTok{": "}\NormalTok{ << }\BuiltInTok{std::}\NormalTok{endl}
\NormalTok{  << (}\DecValTok{3+5}\NormalTok{)*}\DecValTok{20}\NormalTok{ << }\BuiltInTok{std::}\NormalTok{endl;}
\end{Highlighting}
\end{Shaded}

\end{framed}

\end{minipage}
\hfill
\begin{minipage}[t]{.485\linewidth}

\begin{framed}

\begin{verbatim}
A simple operation between 
3
5
20: 
160
\end{verbatim}

\end{framed}

\end{minipage}
\end{minipage}

\vspace{2mm}\noindent\hrulefill{}

\begin{minipage}{\linewidth}\noindent
{\tiny 009.}\\
\begin{minipage}[t]{.485\linewidth}

Did you noticed that we only used a
\VERB|\BuiltInTok{std::}\NormalTok{cout}|?

What is then the output of the code below?

\begin{framed}

\begin{Shaded}
\begin{Highlighting}[]
\BuiltInTok{std::}\NormalTok{cout}
\NormalTok{  << }\StringTok{"A simple operation between "}\NormalTok{ << }\BuiltInTok{std::}\NormalTok{endl;}
\BuiltInTok{std::}\NormalTok{cout << }\DecValTok{3}\NormalTok{ << }\BuiltInTok{std::}\NormalTok{endl;}
\BuiltInTok{std::}\NormalTok{cout << }\DecValTok{5}\NormalTok{ << }\BuiltInTok{std::}\NormalTok{endl;}
\BuiltInTok{std::}\NormalTok{cout << }\DecValTok{20}\NormalTok{ << }\StringTok{": "}\NormalTok{ << }\BuiltInTok{std::}\NormalTok{endl;}
\BuiltInTok{std::}\NormalTok{cout << (}\DecValTok{3+5}\NormalTok{)*}\DecValTok{20}\NormalTok{ << }\BuiltInTok{std::}\NormalTok{endl;}
\end{Highlighting}
\end{Shaded}

\end{framed}

\end{minipage}
\hfill
\begin{minipage}[t]{.485\linewidth}

\begin{framed}

\begin{verbatim}
A simple operation between 
3
5
20: 
160
\end{verbatim}

\end{framed}

Yeah, it's the same as before!

Notice how the semicolon (\texttt{;}) indicates the ending of a
statement in code. The code above could all be written in a single line
(and not in 6 lines) and it would output the same:\footnote{sorry, the
  single line is too long to show in here all at once.}

\begin{framed}

\begin{Shaded}
\begin{Highlighting}[]
\BuiltInTok{std::}\NormalTok{cout << ... << (}\DecValTok{3+5}\NormalTok{)*}\DecValTok{20}\NormalTok{ << }\BuiltInTok{std::}\NormalTok{endl;}
\end{Highlighting}
\end{Shaded}

\end{framed}

Remember every \VERB|\BuiltInTok{std::}\NormalTok{cout}| is always
paired with a semicolon (\texttt{;}) which indicates the ending of its
effects, like a \emph{dot} indicates the ending of a sentence or
paragraph.

\end{minipage}
\end{minipage}

\vspace{2mm}\noindent\hrulefill{}

\begin{minipage}{\linewidth}\noindent
{\tiny 010.}\\
\begin{minipage}[t]{.485\linewidth}

What happens if you try to compile and run this faulty code?

\begin{framed}

\begin{Shaded}
\begin{Highlighting}[]
\BuiltInTok{std::}\NormalTok{cout}
\NormalTok{  << }\StringTok{"A simple operation between "}\NormalTok{ << }\BuiltInTok{std::}\NormalTok{endl;}
\BuiltInTok{std::}\NormalTok{cout << }\DecValTok{3}\NormalTok{ << }\BuiltInTok{std::}\NormalTok{endl;}
\BuiltInTok{std::}\NormalTok{cout << }\DecValTok{5}\NormalTok{ << }\BuiltInTok{std::}\NormalTok{endl}
\BuiltInTok{std::}\NormalTok{cout << }\DecValTok{20}\NormalTok{ << }\StringTok{": "}\NormalTok{ << }\BuiltInTok{std::}\NormalTok{endl;}
\BuiltInTok{std::}\NormalTok{cout << (}\DecValTok{3+5}\NormalTok{)*}\DecValTok{20}\NormalTok{ << }\BuiltInTok{std::}\NormalTok{endl;}
\end{Highlighting}
\end{Shaded}

\end{framed}

\end{minipage}
\hfill
\begin{minipage}[t]{.485\linewidth}

It fails to compile because there is a semicolon missing in the code!

The error shown by the compiler is actually\footnote{Why ``actually''?
  Well, you will find that most of the time errors thrown by the
  compiler are hard to understand. It is often something that we
  programmers need to learn to do. We learn to understand the confusing
  error messages compilers give us.} useful here, it is telling us that
we forgot a \texttt{;}!

\begin{framed}

\begin{verbatim}
:8:30: error: expected ';' after expression
  std::cout << 5 << std::endl
                             ^
                             ;
1 error generated.
\end{verbatim}

\end{framed}

\end{minipage}
\end{minipage}

\vspace{2mm}\noindent\hrulefill{}

\begin{minipage}{\linewidth}\noindent
{\tiny 011.}\\
\begin{minipage}[t]{.485\linewidth}

But what if we want to not input 5 or 20 twice?

What is the output of the code below?

\inlinetodo{remember to explain how to initialize using \texttt{\{\}}}

\begin{framed}

\begin{Shaded}
\begin{Highlighting}[]
\DataTypeTok{int}\NormalTok{ num_1 = }\DecValTok{3}\NormalTok{;}
\DataTypeTok{int}\NormalTok{ num_2 = }\DecValTok{5}\NormalTok{;}
\DataTypeTok{int}\NormalTok{ num_3 = }\DecValTok{20}\NormalTok{;}

\BuiltInTok{std::}\NormalTok{cout}
\NormalTok{  << }\StringTok{"A simple operation between "}
\NormalTok{  << num_1 << }\StringTok{" "}
\NormalTok{  << num_2 << }\StringTok{" "}
\NormalTok{  << num_3 << }\StringTok{": "}
\NormalTok{  << (num_1+num_2)*nu}\VariableTok{m_3}
\NormalTok{  << }\BuiltInTok{std::}\NormalTok{endl;}
\end{Highlighting}
\end{Shaded}

\end{framed}

\end{minipage}
\hfill
\begin{minipage}[t]{.485\linewidth}

The output is:

\begin{framed}

\begin{verbatim}
A simple operation between 3 5 20: 160
\end{verbatim}

\end{framed}

\end{minipage}
\end{minipage}

\vspace{2mm}\noindent\hrulefill{}

\begin{minipage}{\linewidth}\noindent
{\tiny 012.}\\
\begin{minipage}[t]{.485\linewidth}

what is the output if you change the value 5 for 7?

\end{minipage}
\hfill
\begin{minipage}[t]{.485\linewidth}

The output is:

\begin{framed}

\begin{verbatim}
A simple operation between 3 7 20: 200
\end{verbatim}

\end{framed}

\end{minipage}
\end{minipage}

\vspace{2mm}\noindent\hrulefill{}

\begin{minipage}{\linewidth}\noindent
{\tiny 013.}\\
\begin{minipage}[t]{.485\linewidth}

\texttt{num\_1} is a \textbf{variable} and it allows us to save integers
on it, you can try changing it's value for any number between
\(-2147483649\) and \(2147483648\)\footnote{This numbers are based on a
  program compiled for a 32bit computer, the real values dependent
  between different computers.}

What if we put \(-12\) in the variable \texttt{num\_1}?

\end{minipage}
\hfill
\begin{minipage}[t]{.485\linewidth}

The output is:

\begin{framed}

\begin{verbatim}
A simple operation between -12 5 20: -140
\end{verbatim}

\end{framed}

\end{minipage}
\end{minipage}

\vspace{2mm}\noindent\hrulefill{}

\newpage

\hypertarget{interlude-variables}{%
\subsection{Interlude: Variables}\label{interlude-variables}}

Now, it's time to explain what are \textbf{variables} and what happens
when we write \VERB|\DataTypeTok{int}\NormalTok{ name = }\DecValTok{0}|.

\VERB|\DataTypeTok{int}\NormalTok{ name = }\DecValTok{0}| is equivalent
to:\footnote{only for the simplest values \texttt{int}, \texttt{double},
  \ldots{}, but not for objects. Objects are out of the scope of this
  book, but it is important to know they exist.}

\begin{framed}

\begin{Shaded}
\begin{Highlighting}[]
\DataTypeTok{int}\NormalTok{ name;}
\NormalTok{name = }\DecValTok{0}\NormalTok{;}
\end{Highlighting}
\end{Shaded}

\end{framed}

The first instruction \textbf{declares} a space for an \texttt{int} in
memory (RAM memomry).

So, let's study the architecture of computers, mainly RAM and CPU.

\inlinetodo{ADD explanation on the architecture of computers}

\VERB|\DataTypeTok{int}\NormalTok{ name;}| then is telling the compiler
to reserve (\emph{declare}) some space that nobody else should use. This
space can have any value we want. Because this space in memory could
have been used by anybody else in the past, its value is
\emph{nondefined}, meaning that it can be anything. Therefore we use the
next line \VERB|\NormalTok{name = }\DecValTok{0}\NormalTok{;}| to save a
zero in the space declared.

BEWARE! \VERB|\NormalTok{name = }\DecValTok{0}\NormalTok{;}| is NOT an
equation!

I repeat, \VERB|\NormalTok{name = }\DecValTok{0}\NormalTok{;}| is NOT an
equation!, you are \textbf{assigning} a value to a variable, you could
easily assign many different values to a variable, though just the last
one will stay in memory:

\begin{framed}

\begin{Shaded}
\begin{Highlighting}[]
\DataTypeTok{int}\NormalTok{ name;}
\NormalTok{name = }\DecValTok{0}\NormalTok{;}
\NormalTok{name = }\DecValTok{12}\NormalTok{;}
\end{Highlighting}
\end{Shaded}

\end{framed}

The procedure of \emph{declaring} and then \emph{assigning} a value to a
variable is so common that the designers of the language have made a
shortcut:

\begin{framed}

\begin{Shaded}
\begin{Highlighting}[]
\DataTypeTok{int}\NormalTok{ name = }\DecValTok{0}\NormalTok{;}
\end{Highlighting}
\end{Shaded}

\end{framed}

Now, let's go back to the code!

\newpage

\vspace{2mm}\noindent\hrulefill{}

\begin{minipage}{\linewidth}\noindent
{\tiny 014.}\\
\begin{minipage}[t]{.485\linewidth}

What is the output of:

\begin{framed}

\begin{Shaded}
\begin{Highlighting}[]
\DataTypeTok{int}\NormalTok{ var1 = }\DecValTok{6}\NormalTok{;}
\DataTypeTok{int}\NormalTok{ var2 = }\DecValTok{3}\NormalTok{;}
\DataTypeTok{int}\NormalTok{ var3 = }\DecValTok{10}\NormalTok{;}

\BuiltInTok{std::}\NormalTok{cout}
\NormalTok{  << (var1+var2) * var3 - var}\DecValTok{2}
\NormalTok{  << }\BuiltInTok{std::}\NormalTok{endl;}
\end{Highlighting}
\end{Shaded}

\end{framed}

\end{minipage}
\hfill
\begin{minipage}[t]{.485\linewidth}

The output is:

\begin{framed}

\begin{verbatim}
87
\end{verbatim}

\end{framed}

\end{minipage}
\end{minipage}

\vspace{2mm}\noindent\hrulefill{}

\begin{minipage}{\linewidth}\noindent
{\tiny 015.}\\
\begin{minipage}[t]{.485\linewidth}

What is the output of:

\begin{framed}

\begin{Shaded}
\begin{Highlighting}[]
\DataTypeTok{int}\NormalTok{ var1 = }\DecValTok{6}\NormalTok{;}
\DataTypeTok{int}\NormalTok{ var2 = }\DecValTok{3}\NormalTok{;}

\BuiltInTok{std::}\NormalTok{cout}
\NormalTok{  << (var1+var2) * var3 - var}\DecValTok{2}
\NormalTok{  << }\BuiltInTok{std::}\NormalTok{endl;}

\DataTypeTok{int}\NormalTok{ var3 = }\DecValTok{10}\NormalTok{;}
\end{Highlighting}
\end{Shaded}

\end{framed}

\end{minipage}
\hfill
\begin{minipage}[t]{.485\linewidth}

Yeah, it doesn't compile, you are trying to use a variable \emph{before}
declaring it (asking for a space in memory to use it). The compiler
gives you the answer:

\begin{framed}

\begin{verbatim}
:9:20: error: use of undeclared identifier 'var3'
  << (var1+var2) * var3 - var2
                   ^
1 error generated.
\end{verbatim}

\end{framed}

ORDER (of sentences) is key! It is not the same to say ``Peter eats
spaguetti, then Peter clean his teeth'' than ``Peter clean his teeth,
then Peter eats spaguetti''.

A program runs sequentially from the first line of code to the last

\end{minipage}
\end{minipage}

\vspace{2mm}\noindent\hrulefill{}

\begin{minipage}{\linewidth}\noindent
{\tiny 016.}\\
\begin{minipage}[t]{.485\linewidth}

What is the output of:

\begin{framed}

\begin{Shaded}
\begin{Highlighting}[]
\DataTypeTok{int}\NormalTok{ var1 = }\DecValTok{6}\NormalTok{;}
\DataTypeTok{int}\NormalTok{ var2 = }\DecValTok{3}\NormalTok{;}
\DataTypeTok{int}\NormalTok{ var3 = }\DecValTok{10}\NormalTok{;}

\NormalTok{var2 = var1*}\DecValTok{3}\NormalTok{;}

\BuiltInTok{std::}\NormalTok{cout}
\NormalTok{  << (var1+var2) * var3 - var}\DecValTok{2}
\NormalTok{  << }\BuiltInTok{std::}\NormalTok{endl;}
\end{Highlighting}
\end{Shaded}

\end{framed}

\end{minipage}
\hfill
\begin{minipage}[t]{.485\linewidth}

The output is:

\begin{framed}

\begin{verbatim}
222
\end{verbatim}

\end{framed}

We can in fact assign to the variable (at the left of \texttt{=}) any
\VERB|\DataTypeTok{int}| value result of any computation. In this case,
the computation \VERB|\NormalTok{var1*}\DecValTok{3}| is assigned to
\VERB|\NormalTok{var}\DecValTok{2}|

\end{minipage}
\end{minipage}

\vspace{2mm}\noindent\hrulefill{}

\begin{minipage}{\linewidth}\noindent
{\tiny 017.}\\
\begin{minipage}[t]{.485\linewidth}

What is the output of:

\begin{framed}

\begin{Shaded}
\begin{Highlighting}[]
\DataTypeTok{int}\NormalTok{ var1 = }\DecValTok{6}\NormalTok{;}
\BuiltInTok{std::}\NormalTok{cout << var1 << }\StringTok{" "}\NormalTok{;}
\NormalTok{var1 = }\DecValTok{20}\NormalTok{;}
\BuiltInTok{std::}\NormalTok{cout << var1 << }\StringTok{" "}\NormalTok{;}
\NormalTok{var1 = }\DecValTok{-5}\NormalTok{;}
\BuiltInTok{std::}\NormalTok{cout << var1 << }\BuiltInTok{std::}\NormalTok{endl;}
\end{Highlighting}
\end{Shaded}

\end{framed}

\end{minipage}
\hfill
\begin{minipage}[t]{.485\linewidth}

The output is:

\begin{framed}

\begin{verbatim}
6 20 -5
\end{verbatim}

\end{framed}

\end{minipage}
\end{minipage}

\vspace{2mm}\noindent\hrulefill{}

\begin{minipage}{\linewidth}\noindent
{\tiny 018.}\\
\begin{minipage}[t]{.485\linewidth}

What is the output of:

\begin{framed}

\begin{Shaded}
\begin{Highlighting}[]
\DataTypeTok{int}\NormalTok{ var1 = }\DecValTok{6}\NormalTok{;}
\BuiltInTok{std::}\NormalTok{cout << var1 << }\StringTok{" "}\NormalTok{;}
\DataTypeTok{int}\NormalTok{ var1 = }\DecValTok{20}\NormalTok{;}
\BuiltInTok{std::}\NormalTok{cout << var1 << }\StringTok{" "}\NormalTok{;}
\NormalTok{var1 = }\DecValTok{-5}\NormalTok{;}
\BuiltInTok{std::}\NormalTok{cout << var1 << }\BuiltInTok{std::}\NormalTok{endl;}
\end{Highlighting}
\end{Shaded}

\end{framed}

\end{minipage}
\hfill
\begin{minipage}[t]{.485\linewidth}

It doesn't compile because you cannot ask for more space in memory
(declare) with the same name variable, you gotta use a different
name.\footnote{this will be truth until we learn how to ``shadow'' a
  variable name with help of scopes
  \inlinetodo{mm..., doesn't sound that good, rewrite}}

\end{minipage}
\end{minipage}

\vspace{2mm}\noindent\hrulefill{}

\begin{minipage}{\linewidth}\noindent
{\tiny 019.}\\
\begin{minipage}[t]{.485\linewidth}

What is the output of:\footnote{Notice the dots (\texttt{...}) in the
  code above. This dots are just for notation, they aren't meant to be
  written in the source code. The dots represent a division between two
  different parts of code.

  \inlinetodo{Add whole file here}}

\begin{framed}

\begin{Shaded}
\begin{Highlighting}[]
\PreprocessorTok{#include }\ImportTok{<cmath>}
\NormalTok{...}
\DataTypeTok{int}\NormalTok{ var1 = }\DecValTok{6}\NormalTok{;}
\DataTypeTok{int}\NormalTok{ var2 = }\DecValTok{3}\NormalTok{;}
\DataTypeTok{int}\NormalTok{ var3 = var1 * pow(var2, }\DecValTok{3}\NormalTok{);}

\BuiltInTok{std::}\NormalTok{cout << }\StringTok{"var1 * pow(var2, 3) => "}
\NormalTok{          << var3 << }\BuiltInTok{std::}\NormalTok{endl;}
\end{Highlighting}
\end{Shaded}

\end{framed}

\end{minipage}
\hfill
\begin{minipage}[t]{.485\linewidth}

The output is:

\begin{framed}

\begin{verbatim}
var1 * pow(var2, 3) => 162
\end{verbatim}

\end{framed}

Remember that

\begin{Shaded}
\begin{Highlighting}[]
\DataTypeTok{int}\NormalTok{ var3 = var1 * pow(var2, }\DecValTok{3}\NormalTok{);}
\end{Highlighting}
\end{Shaded}

is equivalent to

\begin{Shaded}
\begin{Highlighting}[]
\DataTypeTok{int}\NormalTok{ var3;}
\NormalTok{var3 = var1 * pow(var2, }\DecValTok{3}\NormalTok{);}
\end{Highlighting}
\end{Shaded}

\end{minipage}
\end{minipage}

\vspace{2mm}\noindent\hrulefill{}

\begin{minipage}{\linewidth}\noindent
{\tiny 020.}\\
\begin{minipage}[t]{.485\linewidth}

Till now we've seen just two operations (\texttt{*} and \texttt{+}), but
there are plenty more:

\begin{framed}

\begin{Shaded}
\begin{Highlighting}[]
\DataTypeTok{int}\NormalTok{ var1 = (}\DecValTok{2}\NormalTok{ + }\DecValTok{18}\NormalTok{ - }\DecValTok{6}\NormalTok{ * }\DecValTok{2}\NormalTok{) * }\DecValTok{5}\NormalTok{;}
\DataTypeTok{int}\NormalTok{ var2 = var1 / }\DecValTok{3}\NormalTok{;}
\DataTypeTok{int}\NormalTok{ var3 = var1 % }\DecValTok{3}\NormalTok{;}

\BuiltInTok{std::}\NormalTok{cout << }\StringTok{"var1 => "}\NormalTok{ << var1 << }\BuiltInTok{std::}\NormalTok{endl;}
\BuiltInTok{std::}\NormalTok{cout << }\StringTok{"var2 => "}\NormalTok{ << var2 << }\BuiltInTok{std::}\NormalTok{endl;}
\BuiltInTok{std::}\NormalTok{cout << }\StringTok{"var3 => "}\NormalTok{ << var3 << }\BuiltInTok{std::}\NormalTok{endl;}
\end{Highlighting}
\end{Shaded}

\end{framed}

\end{minipage}
\hfill
\begin{minipage}[t]{.485\linewidth}

The output is:

\begin{framed}

\begin{verbatim}
var1 => 40
var2 => 13
var3 => 1
\end{verbatim}

\end{framed}

You probably know all operations here, but maybe not \texttt{\%}. It is
the \emph{modulus} operation, or residue operation, and it symbolises
the result of the residue of dividing integer numbers.

For example, the result of dividing \(50\) by \(3\) can be written as:

\[ 50 = 3 \times 16 + 2 \]

Where \(50\) is the dividend, \(3\) is the divisor, \(16\) the (integer)
result, and \(2\) the modulus/remainder.

If an integer is divisible by another then the modulus of operating them
must be \(0\), e.g., \(21 = 7 \times 3 + 0\).

\end{minipage}
\end{minipage}

\vspace{2mm}\noindent\hrulefill{}

\begin{minipage}{\linewidth}\noindent
{\tiny 021.}\\
\begin{minipage}[t]{.485\linewidth}

What is the output of:

\begin{framed}

\begin{Shaded}
\begin{Highlighting}[]
\BuiltInTok{std::}\NormalTok{cout << }\DecValTok{8}\NormalTok{ % }\DecValTok{3}\NormalTok{ << }\StringTok{" "}
\NormalTok{          << }\DecValTok{8}\NormalTok{ % }\DecValTok{2}\NormalTok{ << }\StringTok{" "}
\NormalTok{          << }\DecValTok{7}\NormalTok{ % }\DecValTok{2}\NormalTok{ << }\StringTok{" "}
\NormalTok{          << }\DecValTok{17}\NormalTok{ % }\DecValTok{1}\NormalTok{ << }\StringTok{" "}
\NormalTok{          << }\DecValTok{17}\NormalTok{ % }\DecValTok{7}\NormalTok{ << }\StringTok{" "}
\NormalTok{          << }\BuiltInTok{std::}\NormalTok{endl;}
\end{Highlighting}
\end{Shaded}

\end{framed}

\end{minipage}
\hfill
\begin{minipage}[t]{.485\linewidth}

The output is:

\begin{framed}

\begin{verbatim}
2 0 1 0 3 
\end{verbatim}

\end{framed}

\end{minipage}
\end{minipage}

\vspace{2mm}\noindent\hrulefill{}

\begin{minipage}{\linewidth}\noindent
{\tiny 022.}\\
\begin{minipage}[t]{.485\linewidth}

\begin{framed}

\begin{Shaded}
\begin{Highlighting}[]
\BuiltInTok{std::}\NormalTok{cout << }\DecValTok{20}\NormalTok{ - }\DecValTok{6}\NormalTok{ * }\DecValTok{2}\NormalTok{ << }\StringTok{" "}
\NormalTok{          << (}\DecValTok{20}\NormalTok{ - }\DecValTok{6}\NormalTok{) * }\DecValTok{2}\NormalTok{ << }\StringTok{" "}
\NormalTok{          << }\DecValTok{20}\NormalTok{ - (}\DecValTok{6}\NormalTok{ * }\DecValTok{2}\NormalTok{) << }\StringTok{" "}
\NormalTok{          << }\BuiltInTok{std::}\NormalTok{endl;}
\end{Highlighting}
\end{Shaded}

\end{framed}

\end{minipage}
\hfill
\begin{minipage}[t]{.485\linewidth}

\begin{framed}

\begin{verbatim}
8 28 8 
\end{verbatim}

\end{framed}

Notice how \texttt{20\ -\ 6\ *\ 2} does reduce\footnote{or compute} to
\(8\) and not to \(28\)! (i.e.,
\(20-6 \times 2 = 20-(6 \times 2) \neq (20-6) \times 2\)). Each operator
has a specific precedence that indicates if it must be applied before
another operator, \texttt{*} for example has a higher precedence than
\texttt{+}.

\end{minipage}
\end{minipage}

\vspace{2mm}\noindent\hrulefill{}

\inlinetodo{add exercises about all of this}

\inlinetodo{add one exercise that ask students to write a program with a single
compilation error and ask another to find it and correct it}

\vspace{2mm}\noindent\hrulefill{}

\begin{minipage}{\linewidth}\noindent
{\tiny 023.}\\
\begin{minipage}[t]{.485\linewidth}

What is the output of:\footnote{\VERB|\DataTypeTok{double}| allows us to
  declare ``real'' numbers (they are actually rational). ANd we can
  operate with them as we did with \VERB|\DataTypeTok{int}|'s}

\begin{framed}

\begin{Shaded}
\begin{Highlighting}[]
\DataTypeTok{double}\NormalTok{ acction   = }\FloatTok{9.8}\NormalTok{; }\CommentTok{// m/s^2 acceleration}
\DataTypeTok{double}\NormalTok{ mass      = }\DecValTok{3}\NormalTok{;   }\CommentTok{// kg    mass}
\DataTypeTok{double}\NormalTok{ initial_v = }\DecValTok{10}\NormalTok{;  }\CommentTok{// m/s   ini. velocity}
\DataTypeTok{double}\NormalTok{ time      = }\FloatTok{2.3}\NormalTok{; }\CommentTok{// s     time passed}

\DataTypeTok{double}\NormalTok{ final_v = initial_v + acction * time;}

\BuiltInTok{std::}\NormalTok{cout << }\StringTok{"Velocity after "}\NormalTok{ << time}
\NormalTok{          << }\StringTok{"s is: "}\NormalTok{ << final_v << }\StringTok{"m/s"}
\NormalTok{          << }\BuiltInTok{std::}\NormalTok{endl;}

\DataTypeTok{double}\NormalTok{ momentum = mass * final_v;}

\BuiltInTok{std::}\NormalTok{cout << }\StringTok{"Momentum after "}\NormalTok{ << time}
\NormalTok{          << }\StringTok{"s is: "}\NormalTok{ << momentum << }\StringTok{"kg*m/s"}
\NormalTok{          << }\BuiltInTok{std::}\NormalTok{endl;}
\end{Highlighting}
\end{Shaded}

\end{framed}

\end{minipage}
\hfill
\begin{minipage}[t]{.485\linewidth}

The output is:

\begin{framed}

\begin{verbatim}
Velocity after 2.3s is: 32.54m/s
Momentum after 2.3s is: 97.62kg*m/s
\end{verbatim}

\end{framed}

We are using here the equation \(v = u + a*t\) to determine the final
velocity of an object (in a line) after \(2.3 s\). The object starts
with a velocity \(2.3 m/s\), has a constant acceleration of
\(9.8 m/s^2\)\footnote{free fall ;)}, and we know the weight of the
object, so we can calculate too its momentum.

\end{minipage}
\end{minipage}

\vspace{2mm}\noindent\hrulefill{}

\begin{minipage}{\linewidth}\noindent
{\tiny 024.}\\
\begin{minipage}[t]{.485\linewidth}

What is the output of:

\begin{framed}

\begin{Shaded}
\begin{Highlighting}[]
\DataTypeTok{int}\NormalTok{ var1 = }\DecValTok{8}\NormalTok{;}
\ControlFlowTok{if}\NormalTok{ (var1 < }\DecValTok{10}\NormalTok{) \{}
  \BuiltInTok{std::}\NormalTok{cout << }\StringTok{"var1 smaller than 10"}
\NormalTok{            << }\BuiltInTok{std::}\NormalTok{endl;}
\NormalTok{\} }\ControlFlowTok{else}\NormalTok{ \{}
  \BuiltInTok{std::}\NormalTok{cout}
\NormalTok{    << }\StringTok{"var1 greater than or equal to 10"}
\NormalTok{    << }\BuiltInTok{std::}\NormalTok{endl;}
\NormalTok{\}}
\end{Highlighting}
\end{Shaded}

\end{framed}

\end{minipage}
\hfill
\begin{minipage}[t]{.485\linewidth}

The output is:

\begin{framed}

\begin{verbatim}
var1 smaller than 10
\end{verbatim}

\end{framed}

\end{minipage}
\end{minipage}

\vspace{2mm}\noindent\hrulefill{}

\begin{minipage}{\linewidth}\noindent
{\tiny 025.}\\
\begin{minipage}[t]{.485\linewidth}

What is the output of the code above if we change \texttt{var1}'s
assignment from \texttt{8} to \texttt{19}?

\end{minipage}
\hfill
\begin{minipage}[t]{.485\linewidth}

The output is:

\begin{framed}

\begin{verbatim}
var1 greater than or equal to 10
\end{verbatim}

\end{framed}

Precisely, the second line runs but not the first.

The structure:

\begin{framed}

\begin{Shaded}
\begin{Highlighting}[]
\ControlFlowTok{if}\NormalTok{ (statement) \{}
  \KeywordTok{true}\NormalTok{ branch}
\NormalTok{\} }\ControlFlowTok{else}\NormalTok{ \{}
  \KeywordTok{false}\NormalTok{ branch}
\NormalTok{\}}
\end{Highlighting}
\end{Shaded}

\end{framed}

runs the \texttt{true\ branch} if the statement \texttt{statement}
evaluates to \texttt{true} otherwise it runs the \texttt{false\ branch}.

\end{minipage}
\end{minipage}

\vspace{2mm}\noindent\hrulefill{}

\begin{minipage}{\linewidth}\noindent
{\tiny 026.}\\
\begin{minipage}[t]{.485\linewidth}

What about the output of

\begin{framed}

\begin{Shaded}
\begin{Highlighting}[]
\DataTypeTok{int}\NormalTok{ var1 = }\DecValTok{10}\NormalTok{;}
\ControlFlowTok{if}\NormalTok{ (var1 == }\DecValTok{7}\NormalTok{ + }\DecValTok{3}\NormalTok{) \{}
  \BuiltInTok{std::}\NormalTok{cout << }\StringTok{"var1 is equal to 7 + 3"}
\NormalTok{            << }\BuiltInTok{std::}\NormalTok{endl;}
\NormalTok{\} }\ControlFlowTok{else}\NormalTok{ \{}
  \BuiltInTok{std::}\NormalTok{cout << }\StringTok{"var1 does not equal 10"}
\NormalTok{    << }\BuiltInTok{std::}\NormalTok{endl;}
\NormalTok{\}}
\end{Highlighting}
\end{Shaded}

\end{framed}

\end{minipage}
\hfill
\begin{minipage}[t]{.485\linewidth}

Well, that's easy:

\begin{framed}

\begin{verbatim}
var1 is equal to 7 + 3
\end{verbatim}

\end{framed}

Note, \texttt{==} is the equality operation\footnote{take care, don't
  confuse it with the assignment operator \texttt{=}!}, and it compares
any two statements as \texttt{var1} or \texttt{3+2*12} for equality.

As you expect it, there are many operations to compare between different
statements, these are \texttt{==}, \texttt{\textless{}=},
\texttt{\textgreater{}=}, \texttt{\textless{}}, \texttt{\textgreater{}},
and \texttt{!=}.

\end{minipage}
\end{minipage}

\vspace{2mm}\noindent\hrulefill{}

\begin{minipage}{\linewidth}\noindent
{\tiny 027.}\\
\begin{minipage}[t]{.485\linewidth}

What is the output of:

\begin{framed}

\begin{Shaded}
\begin{Highlighting}[]
\DataTypeTok{int}\NormalTok{ var1 = }\DecValTok{2}\NormalTok{*}\DecValTok{2+2}\NormalTok{;}
\ControlFlowTok{if}\NormalTok{ (var1 != }\DecValTok{7}\NormalTok{ + }\DecValTok{3}\NormalTok{) \{}
  \BuiltInTok{std::}\NormalTok{cout << }\StringTok{"2*2+2 != 7+3"}\NormalTok{ << }\BuiltInTok{std::}\NormalTok{endl;}
\NormalTok{\} }\ControlFlowTok{else}\NormalTok{ \{}
  \BuiltInTok{std::}\NormalTok{cout << }\StringTok{"2*2+2 == 7+3"}\NormalTok{ << }\BuiltInTok{std::}\NormalTok{endl;}
\NormalTok{\}}
\end{Highlighting}
\end{Shaded}

\end{framed}

\end{minipage}
\hfill
\begin{minipage}[t]{.485\linewidth}

The output is:

\begin{framed}

\begin{verbatim}
2*2+2 != 7+3
\end{verbatim}

\end{framed}

Yeah, \texttt{!=} is the unequality operator.

\end{minipage}
\end{minipage}

\vspace{2mm}\noindent\hrulefill{}

\begin{minipage}{\linewidth}\noindent
{\tiny 028.}\\
\begin{minipage}[t]{.485\linewidth}

What is the output of:

\begin{framed}

\begin{Shaded}
\begin{Highlighting}[]
\DataTypeTok{int}\NormalTok{ var1 = }\DecValTok{2}\NormalTok{*}\DecValTok{2+20}\NormalTok{;}
\ControlFlowTok{if}\NormalTok{ (var1 >= }\DecValTok{7}\NormalTok{ + }\DecValTok{3}\NormalTok{) \{}
  \BuiltInTok{std::}\NormalTok{cout << }\StringTok{"i never run :("}\NormalTok{ << }\BuiltInTok{std::}\NormalTok{endl;}
\NormalTok{\} }\ControlFlowTok{else}\NormalTok{ \{}
  \ControlFlowTok{if}\NormalTok{ (var1 > }\DecValTok{23}\NormalTok{) \{}
    \BuiltInTok{std::}\NormalTok{cout << }\StringTok{"hey!!"}\NormalTok{ << }\BuiltInTok{std::}\NormalTok{endl;}
\NormalTok{  \} }\ControlFlowTok{else}\NormalTok{ \{}
    \CommentTok{// nothing in this branch}
\NormalTok{  \}}
\NormalTok{\}}
\end{Highlighting}
\end{Shaded}

\end{framed}

\end{minipage}
\hfill
\begin{minipage}[t]{.485\linewidth}

The output is:

\begin{framed}

\begin{verbatim}
i never run :(
\end{verbatim}

\end{framed}

Usually, if we only care about the true branch of an \texttt{if}
statement, then we simply ignore it. The code at left is equivalent then
to:

\begin{framed}

\begin{Shaded}
\begin{Highlighting}[]
\DataTypeTok{int}\NormalTok{ var1 = }\DecValTok{2}\NormalTok{*}\DecValTok{2+20}\NormalTok{;}
\ControlFlowTok{if}\NormalTok{ (var1 >= }\DecValTok{7}\NormalTok{ + }\DecValTok{3}\NormalTok{) \{}
  \BuiltInTok{std::}\NormalTok{cout << }\StringTok{"i never run :("}\NormalTok{ << }\BuiltInTok{std::}\NormalTok{endl;}
\NormalTok{\} }\ControlFlowTok{else}\NormalTok{ \{}
  \ControlFlowTok{if}\NormalTok{ (var1 > }\DecValTok{23}\NormalTok{) \{}
    \BuiltInTok{std::}\NormalTok{cout << }\StringTok{"hey!!"}\NormalTok{ << }\BuiltInTok{std::}\NormalTok{endl;}
\NormalTok{  \}}
\NormalTok{\}}
\end{Highlighting}
\end{Shaded}

\end{framed}

\end{minipage}
\end{minipage}

\vspace{2mm}\noindent\hrulefill{}

\begin{minipage}{\linewidth}\noindent
{\tiny 029.}\\
\begin{minipage}[t]{.485\linewidth}

It's possible to put more than one statement inside the \texttt{if}
statement. For example:

\begin{framed}

\begin{Shaded}
\begin{Highlighting}[]
\DataTypeTok{int}\NormalTok{ var1 = }\DecValTok{2}\NormalTok{*}\DecValTok{2+20}\NormalTok{;}
\ControlFlowTok{if}\NormalTok{ (var1 >= }\DecValTok{7}\NormalTok{ + }\DecValTok{3}\NormalTok{) \{}
  \BuiltInTok{std::}\NormalTok{cout}
\NormalTok{    << }\StringTok{"you are our visitor number 3889"}
\NormalTok{    << }\BuiltInTok{std::}\NormalTok{endl;}
  \DataTypeTok{int}\NormalTok{ magicnumber = }\DecValTok{999999}\NormalTok{;}
  \BuiltInTok{std::}\NormalTok{cout}
\NormalTok{    << }\StringTok{"and your our winner! please "}
\NormalTok{    << }\StringTok{"deposit "}\NormalTok{ << magicnumber}
\NormalTok{    << }\StringTok{" into our account and you will "}
\NormalTok{    << }\StringTok{" receive 20x what you deposited "}
\NormalTok{    << }\BuiltInTok{std::}\NormalTok{endl;}
\NormalTok{\}}
\BuiltInTok{std::}\NormalTok{cout}
\NormalTok{  << }\StringTok{"Welcome to e-safe-comerce"}
\NormalTok{  << }\BuiltInTok{std::}\NormalTok{endl;}
\end{Highlighting}
\end{Shaded}

\end{framed}

\end{minipage}
\hfill
\begin{minipage}[t]{.485\linewidth}

The output as you expected.

\begin{framed}

\begin{verbatim}
you are our visitor number 3889
and your our winner! please deposit 999999 into our account and you will  receive 20x what you deposited 
Welcome to e-safe-comerce
\end{verbatim}

\end{framed}

\end{minipage}
\end{minipage}

\vspace{2mm}\noindent\hrulefill{}

\begin{minipage}{\linewidth}\noindent
{\tiny 030.}\\
\begin{minipage}[t]{.485\linewidth}

\begin{framed}

\begin{Shaded}
\begin{Highlighting}[]
\DataTypeTok{int}\NormalTok{ magicnumber = }\DecValTok{-2}\NormalTok{;}
\ControlFlowTok{if}\NormalTok{ (}\DecValTok{2}\NormalTok{!=}\DecValTok{3}\NormalTok{) \{}
\NormalTok{  magicnumber = }\DecValTok{0}\NormalTok{;}
\NormalTok{\} }\ControlFlowTok{else}\NormalTok{ \{}
\NormalTok{  magicnumber = }\DecValTok{42}\NormalTok{;}
\NormalTok{\}}
\BuiltInTok{std::}\NormalTok{cout << }\StringTok{"Every integer is a divisor of "}
\NormalTok{          << magicnumber << }\BuiltInTok{std::}\NormalTok{endl;}
\end{Highlighting}
\end{Shaded}

\end{framed}

\end{minipage}
\hfill
\begin{minipage}[t]{.485\linewidth}

Of course, \(0\) can be divided by any number, because it can be written
as \(0 = k*n\) where \(k=0\) and \(n\) is an arbitrary number.

\begin{framed}

\begin{verbatim}
Every integer is a divisor of 0
\end{verbatim}

\end{framed}

\end{minipage}
\end{minipage}

\vspace{2mm}\noindent\hrulefill{}

\begin{minipage}{\linewidth}\noindent
{\tiny 031.}\\
\begin{minipage}[t]{.485\linewidth}

What is the output of:

\begin{framed}

\begin{Shaded}
\begin{Highlighting}[]
\ControlFlowTok{if}\NormalTok{ (}\DecValTok{2}\NormalTok{!=}\DecValTok{3}\NormalTok{) \{}
  \DataTypeTok{int}\NormalTok{ magicnumber = }\DecValTok{0}\NormalTok{;}
\NormalTok{\} }\ControlFlowTok{else}\NormalTok{ \{}
  \DataTypeTok{int}\NormalTok{ magicnumber = }\DecValTok{42}\NormalTok{;}
\NormalTok{\}}
\BuiltInTok{std::}\NormalTok{cout << }\StringTok{"Every integer is a divisor of "}
\NormalTok{          << magicnumber << }\BuiltInTok{std::}\NormalTok{endl;}
\end{Highlighting}
\end{Shaded}

\end{framed}

\end{minipage}
\hfill
\begin{minipage}[t]{.485\linewidth}

It doesn't compile! You wanna know why? Well, keep guessing with the
following exercises.

\end{minipage}
\end{minipage}

\vspace{2mm}\noindent\hrulefill{}

\begin{minipage}{\linewidth}\noindent
{\tiny 032.}\\
\begin{minipage}[t]{.485\linewidth}

Does this compile? If yes, then what is its output?

\begin{framed}

\begin{Shaded}
\begin{Highlighting}[]
\DataTypeTok{int}\NormalTok{ magicnumber = }\DecValTok{-2}\NormalTok{;}
\ControlFlowTok{if}\NormalTok{ (}\DecValTok{2}\NormalTok{!=}\DecValTok{3}\NormalTok{) \{}
  \DataTypeTok{int}\NormalTok{ magicnumber = }\DecValTok{0}\NormalTok{;}
\NormalTok{\} }\ControlFlowTok{else}\NormalTok{ \{}
  \DataTypeTok{int}\NormalTok{ magicnumber = }\DecValTok{42}\NormalTok{;}
\NormalTok{\}}
\BuiltInTok{std::}\NormalTok{cout << }\StringTok{"Every integer is a divisor of "}
\NormalTok{          << magicnumber << }\BuiltInTok{std::}\NormalTok{endl;}
\end{Highlighting}
\end{Shaded}

\end{framed}

It does compile, and its output is:

\end{minipage}
\hfill
\begin{minipage}[t]{.485\linewidth}

\begin{framed}

\begin{verbatim}
Every integer is a divisor of -2
\end{verbatim}

\end{framed}

But wait, what? -2 is not always divisible by anyother integer!

\end{minipage}
\end{minipage}

\vspace{2mm}\noindent\hrulefill{}

\begin{minipage}{\linewidth}\noindent
{\tiny 033.}\\
\begin{minipage}[t]{.485\linewidth}

Maybe, \(2\) and \(3\) are really the same thing.

\begin{framed}

\begin{Shaded}
\begin{Highlighting}[]
\DataTypeTok{int}\NormalTok{ magicnumber = }\DecValTok{-2}\NormalTok{;}
\ControlFlowTok{if}\NormalTok{ (}\DecValTok{2}\NormalTok{==}\DecValTok{3}\NormalTok{) \{}
  \DataTypeTok{int}\NormalTok{ magicnumber = }\DecValTok{0}\NormalTok{;}
\NormalTok{\} }\ControlFlowTok{else}\NormalTok{ \{}
  \DataTypeTok{int}\NormalTok{ magicnumber = }\DecValTok{42}\NormalTok{;}
\NormalTok{\}}
\BuiltInTok{std::}\NormalTok{cout << }\StringTok{"Every integer is a divisor of "}
\NormalTok{          << magicnumber << }\BuiltInTok{std::}\NormalTok{endl;}
\end{Highlighting}
\end{Shaded}

\end{framed}

\end{minipage}
\hfill
\begin{minipage}[t]{.485\linewidth}

The output is:

\begin{framed}

\begin{verbatim}
Every integer is a divisor of -2
\end{verbatim}

\end{framed}

But is the same, why? What have we changed from before? What is the
difference with the code that gives us a zero?

\end{minipage}
\end{minipage}

\vspace{2mm}\noindent\hrulefill{}

\begin{minipage}{\linewidth}\noindent
{\tiny 034.}\\
\begin{minipage}[t]{.485\linewidth}

Ok, let's stop with the silliness and try with an example that actually
give us some clue of the situation.

\begin{framed}

\begin{Shaded}
\begin{Highlighting}[]
\DataTypeTok{int}\NormalTok{ num = }\DecValTok{20}\NormalTok{;}
\BuiltInTok{std::}\NormalTok{cout << num << }\StringTok{" "}\NormalTok{;}
\NormalTok{\{}
  \DataTypeTok{int}\NormalTok{ num = }\DecValTok{42}\NormalTok{;}
  \BuiltInTok{std::}\NormalTok{cout << num << }\StringTok{" "}\NormalTok{;}
\NormalTok{  num = }\DecValTok{3}\NormalTok{;}
  \BuiltInTok{std::}\NormalTok{cout << num << }\StringTok{" "}\NormalTok{;}
\NormalTok{\}}
\BuiltInTok{std::}\NormalTok{cout << num << }\StringTok{" "}\NormalTok{;}
\NormalTok{num = }\DecValTok{0}\NormalTok{;}
\BuiltInTok{std::}\NormalTok{cout << num << }\BuiltInTok{std::}\NormalTok{endl;}
\end{Highlighting}
\end{Shaded}

\end{framed}

\end{minipage}
\hfill
\begin{minipage}[t]{.485\linewidth}

Well, the code inside the brackets (\texttt{\{\}}) acts as if it was
being run alone without the intervention of the code of the outside.

\begin{framed}

\begin{verbatim}
20 42 3 20 0
\end{verbatim}

\end{framed}

It is effectively as if this was the code being run:

\begin{framed}

\begin{Shaded}
\begin{Highlighting}[]
\DataTypeTok{int}\NormalTok{ num = }\DecValTok{20}\NormalTok{;}
\BuiltInTok{std::}\NormalTok{cout << num << }\StringTok{" "}\NormalTok{;}
\NormalTok{\{}
  \DataTypeTok{int}\NormalTok{ var = }\DecValTok{42}\NormalTok{;}
  \BuiltInTok{std::}\NormalTok{cout << var << }\StringTok{" "}\NormalTok{;}
\NormalTok{  var = }\DecValTok{3}\NormalTok{;}
  \BuiltInTok{std::}\NormalTok{cout << var << }\StringTok{" "}\NormalTok{;}
\NormalTok{\}}
\BuiltInTok{std::}\NormalTok{cout << num << }\StringTok{" "}\NormalTok{;}
\NormalTok{num = }\DecValTok{0}\NormalTok{;}
\BuiltInTok{std::}\NormalTok{cout << num << }\BuiltInTok{std::}\NormalTok{endl;}
\end{Highlighting}
\end{Shaded}

\end{framed}

\end{minipage}
\end{minipage}

\vspace{2mm}\noindent\hrulefill{}

\begin{minipage}{\linewidth}\noindent
{\tiny 035.}\\
\begin{minipage}[t]{.485\linewidth}

Any time we enclose code between brackets (\texttt{\{\}}) we are
defining a new \textbf{scope}. Any variable we declare inside a scope
lives only in that scope, the variable \emph{dies} once the scope is
closed, is this the reason why this code

\begin{framed}

\begin{Shaded}
\begin{Highlighting}[]
\NormalTok{\{}
  \DataTypeTok{int}\NormalTok{ num = }\DecValTok{42}\NormalTok{;}
  \BuiltInTok{std::}\NormalTok{cout << num << }\StringTok{" "}\NormalTok{;}
\NormalTok{\}}
\BuiltInTok{std::}\NormalTok{cout << num << }\BuiltInTok{std::}\NormalTok{endl;}
\end{Highlighting}
\end{Shaded}

\end{framed}

doesn't compile. There is no \VERB|\NormalTok{num}| variable in the
bigger scope when it wants to show it.

A rule about scopes is that they can access to variables from the
outside, scopes that enclose them. For example this code, does indeed
compiles:

\begin{framed}

\begin{Shaded}
\begin{Highlighting}[]
\DataTypeTok{int}\NormalTok{ num = }\DecValTok{42}\NormalTok{;}
\BuiltInTok{std::}\NormalTok{cout << num << }\StringTok{" "}\NormalTok{;}
\NormalTok{\{}
\NormalTok{  num = }\DecValTok{1}\NormalTok{;}
  \BuiltInTok{std::}\NormalTok{cout << num << }\StringTok{" "}\NormalTok{;}
\NormalTok{\}}
\BuiltInTok{std::}\NormalTok{cout << num << }\BuiltInTok{std::}\NormalTok{endl;}
\end{Highlighting}
\end{Shaded}

\end{framed}

What is its output?

\end{minipage}
\hfill
\begin{minipage}[t]{.485\linewidth}

You guessed it right! Given that \VERB|\NormalTok{num}| is a variable
outside the inner scope, the inner scope can read and modify the
variable content.

\begin{framed}

\begin{verbatim}
42 1 1
\end{verbatim}

\end{framed}

\end{minipage}
\end{minipage}

\vspace{2mm}\noindent\hrulefill{}

\begin{minipage}{\linewidth}\noindent
{\tiny 036.}\\
\begin{minipage}[t]{.485\linewidth}

What is the output of:

\begin{framed}

\begin{Shaded}
\begin{Highlighting}[]
\DataTypeTok{int}\NormalTok{ num = }\DecValTok{42}\NormalTok{;}
\BuiltInTok{std::}\NormalTok{cout << num << }\BuiltInTok{std::}\NormalTok{endl;}
\NormalTok{\{}
  \DataTypeTok{int}\NormalTok{ num = }\DecValTok{1}\NormalTok{;}
  \BuiltInTok{std::}\NormalTok{cout << num << }\BuiltInTok{std::}\NormalTok{endl;}
\NormalTok{\}}
\BuiltInTok{std::}\NormalTok{cout << num << }\BuiltInTok{std::}\NormalTok{endl;}
\NormalTok{\{}
\NormalTok{  num = }\DecValTok{1}\NormalTok{;}
  \BuiltInTok{std::}\NormalTok{cout << num << }\BuiltInTok{std::}\NormalTok{endl;}
\NormalTok{\}}
\BuiltInTok{std::}\NormalTok{cout << num << }\BuiltInTok{std::}\NormalTok{endl;}
\end{Highlighting}
\end{Shaded}

\end{framed}

\end{minipage}
\hfill
\begin{minipage}[t]{.485\linewidth}

\begin{framed}

\begin{verbatim}
42
1
42
1
1
\end{verbatim}

\end{framed}

Right, in the first inner scope, we declare a new space and shadow the
access to the outside variable \VERB|\NormalTok{num}|, and in the other
inner scope, we use the variable accessible from the outside scope.

\end{minipage}
\end{minipage}

\vspace{2mm}\noindent\hrulefill{}

\begin{minipage}{\linewidth}\noindent
{\tiny 037.}\\
\begin{minipage}[t]{.485\linewidth}

What is the ouput of:\footnote{yet again another type of data.
  \VERB|\DataTypeTok{float}|s are like \VERB|\DataTypeTok{double}|s but
  they represent numbers with less precision. Well I haven't talked what
  precision is, forgive me, for the time being just assume that using
  \VERB|\DataTypeTok{double}| in your code is better than using
  \VERB|\DataTypeTok{float}|.
  \inlinetodo{explain at some point what do you mean by precision, maybe another interlude
    could be helpful here}}

\begin{framed}

\begin{Shaded}
\begin{Highlighting}[]
\DataTypeTok{int}\NormalTok{ a = }\DecValTok{20}\NormalTok{;}
\DataTypeTok{int}\NormalTok{ b = }\DecValTok{21}\NormalTok{;}
\ControlFlowTok{if}\NormalTok{ (b-a > }\DecValTok{0}\NormalTok{) \{}
  \DataTypeTok{float}\NormalTok{ num = }\FloatTok{-12.2}\NormalTok{;}
  \BuiltInTok{std::}\NormalTok{cout << num * }\DecValTok{3}\NormalTok{ << }\BuiltInTok{std::}\NormalTok{endl;}
\NormalTok{\} }\ControlFlowTok{else}\NormalTok{ \{}
  \DataTypeTok{double}\NormalTok{ num = }\FloatTok{-12.2}\NormalTok{;}
  \BuiltInTok{std::}\NormalTok{cout << num * }\DecValTok{-3}\NormalTok{ << }\BuiltInTok{std::}\NormalTok{endl;}
\NormalTok{\}}
\end{Highlighting}
\end{Shaded}

\end{framed}

\end{minipage}
\hfill
\begin{minipage}[t]{.485\linewidth}

The output is:

\begin{framed}

\begin{verbatim}
-36.6
\end{verbatim}

\end{framed}

\end{minipage}
\end{minipage}

\vspace{2mm}\noindent\hrulefill{}

\begin{minipage}{\linewidth}\noindent
{\tiny 038.}\\
\begin{minipage}[t]{.485\linewidth}

What does this output:

\begin{framed}

\begin{Shaded}
\begin{Highlighting}[]
\DataTypeTok{int}\NormalTok{ n = }\DecValTok{15}\NormalTok{;}
\DataTypeTok{bool}\NormalTok{ either = n<}\DecValTok{3}\NormalTok{;}
\ControlFlowTok{if}\NormalTok{ (either) \{}
  \BuiltInTok{std::}\NormalTok{cout << }\StringTok{":P"}\NormalTok{ << }\BuiltInTok{std::}\NormalTok{endl;}
\NormalTok{\} }\ControlFlowTok{else}\NormalTok{ \{}
  \BuiltInTok{std::}\NormalTok{cout << }\StringTok{":("}\NormalTok{ << }\BuiltInTok{std::}\NormalTok{endl;}
\NormalTok{\}}
\end{Highlighting}
\end{Shaded}

\end{framed}

\end{minipage}
\hfill
\begin{minipage}[t]{.485\linewidth}

The output is:

\begin{framed}

\begin{verbatim}
:(
\end{verbatim}

\end{framed}

with \VERB|\DataTypeTok{bool}|\footnote{True and False are the only two
  possible values that a statement can take in traditional logic, and so
  it does for us, there are only two options for bool. But modern
  computers are build on blocks of 32 or 64 bits and operations,
  therefore, with values of 32 and 64 bits are cheap to perform.
  Operations on single bits are not simple when you manipulate multiple
  bits, it is slightly more expensive to use a single bit to represent
  truth or false. Compilers usually use a block of memory (32 or 64
  bits) to represent a boolean, the convention is for zero (all 32-64
  bits in zero) to be false and anything else (eg, all bits in zero, or
  only one bit in zero, etc) to be true.} we tell the compiler that it
should interprete \texttt{either} as a boolean (either
\VERB|\KeywordTok{true}| or \VERB|\KeywordTok{false}|).

This means that you can in fact store integer values inside a bool
(i.e., \texttt{bool\ bad\ =\ 23;}) but it's consider bad practice and
may result in undefined behaivor.

\end{minipage}
\end{minipage}

\vspace{2mm}\noindent\hrulefill{}

\begin{minipage}{\linewidth}\noindent
{\tiny 039.}\\
\begin{minipage}[t]{.485\linewidth}

Let's take a look at another type of variable,
\VERB|\DataTypeTok{char}|:\footnote{you can ignore the weird
  \VERB|\NormalTok{(}\DataTypeTok{int}\NormalTok{)}| thing for now, it
  is called \textbf{cast} if you are curious, we will get to them later
  on.}

\begin{framed}

\begin{Shaded}
\begin{Highlighting}[]
\DataTypeTok{char}\NormalTok{ apples = }\DecValTok{23}\NormalTok{;}
\BuiltInTok{std::}\NormalTok{cout << }\StringTok{"A small integer: "}\NormalTok{ << (}\DataTypeTok{int}\NormalTok{)apples}
\NormalTok{          << }\BuiltInTok{std::}\NormalTok{endl;}
\end{Highlighting}
\end{Shaded}

\end{framed}

\end{minipage}
\hfill
\begin{minipage}[t]{.485\linewidth}

The output isn't surprising:

\begin{framed}

\begin{verbatim}
A small integer: 23
\end{verbatim}

\end{framed}

\VERB|\DataTypeTok{char}| may not seem different to
\VERB|\DataTypeTok{int}|, but it is. \texttt{int}, dependending on the
system, has a size of 32 or 64, but \texttt{char} has always the same
size 8 bits. With 8 bits we can represent \(2^8\) different states, that
is 256 different numbers.

\end{minipage}
\end{minipage}

\vspace{2mm}\noindent\hrulefill{}

\begin{minipage}{\linewidth}\noindent
{\tiny 040.}\\
\begin{minipage}[t]{.485\linewidth}

What is the output of:

\begin{framed}

\begin{Shaded}
\begin{Highlighting}[]
\DataTypeTok{int}\NormalTok{ i = }\DecValTok{2}\NormalTok{;}
\BuiltInTok{std::}\NormalTok{cout << i << }\StringTok{" "}\NormalTok{;}
\NormalTok{i = i + }\DecValTok{1}\NormalTok{;}
\BuiltInTok{std::}\NormalTok{cout << i << }\StringTok{" "}\NormalTok{;}
\NormalTok{i = i + }\DecValTok{1}\NormalTok{;}
\BuiltInTok{std::}\NormalTok{cout << i << }\StringTok{" "}\NormalTok{;}
\NormalTok{i = i * }\DecValTok{3}\NormalTok{;}
\BuiltInTok{std::}\NormalTok{cout << i << }\BuiltInTok{std::}\NormalTok{endl;}
\end{Highlighting}
\end{Shaded}

\end{framed}

\end{minipage}
\hfill
\begin{minipage}[t]{.485\linewidth}

The output is:

\begin{framed}

\begin{verbatim}
2 3 4 12
\end{verbatim}

\end{framed}

\end{minipage}
\end{minipage}

\vspace{2mm}\noindent\hrulefill{}

\begin{minipage}{\linewidth}\noindent
{\tiny 041.}\\
\begin{minipage}[t]{.485\linewidth}

Every compiler may define the size of \texttt{int}, \texttt{char},
\texttt{double}, \ldots{}, differently depending on the architecture. If
you want to know how many bytes\footnote{one byte is 8 bits} are
assigned to any variable type, you can use \VERB|\KeywordTok{sizeof}|.
An example of use:

\begin{framed}

\begin{Shaded}
\begin{Highlighting}[]
\BuiltInTok{std::}\NormalTok{cout << }\StringTok{"A char   is "}\NormalTok{ << }\KeywordTok{sizeof}\NormalTok{(}\DataTypeTok{char}\NormalTok{)}
\NormalTok{          << }\StringTok{" bytes"}\NormalTok{ << }\BuiltInTok{std::}\NormalTok{endl;}
\BuiltInTok{std::}\NormalTok{cout << }\StringTok{"A int    is "}\NormalTok{ << }\KeywordTok{sizeof}\NormalTok{(}\DataTypeTok{int}\NormalTok{)}
\NormalTok{          << }\StringTok{" bytes"}\NormalTok{ << }\BuiltInTok{std::}\NormalTok{endl;}
\BuiltInTok{std::}\NormalTok{cout << }\StringTok{"A double is "}\NormalTok{ << }\KeywordTok{sizeof}\NormalTok{(}\DataTypeTok{double}\NormalTok{)}
\NormalTok{          << }\StringTok{" bytes"}\NormalTok{ << }\BuiltInTok{std::}\NormalTok{endl;}
\BuiltInTok{std::}\NormalTok{cout << }\StringTok{"A float  is "}\NormalTok{ << }\KeywordTok{sizeof}\NormalTok{(}\DataTypeTok{float}\NormalTok{)}
\NormalTok{          << }\StringTok{" bytes"}\NormalTok{ << }\BuiltInTok{std::}\NormalTok{endl;}
\BuiltInTok{std::}\NormalTok{cout << }\StringTok{"A bool   is "}\NormalTok{ << }\KeywordTok{sizeof}\NormalTok{(}\DataTypeTok{bool}\NormalTok{)}
\NormalTok{          << }\StringTok{" bytes"}\NormalTok{ << }\BuiltInTok{std::}\NormalTok{endl;}
\end{Highlighting}
\end{Shaded}

\end{framed}

(Hint: if a byte is 8 bits, a \texttt{char} is 8 bits, how many bytes
are a \texttt{char}?)

\end{minipage}
\hfill
\begin{minipage}[t]{.485\linewidth}

This may look different in your computer, but mine runs on 64 bits,
therefore \texttt{double} has 64 bits and \texttt{int} half of that.

\begin{framed}

\begin{verbatim}
A char   is 1 bytes
A int    is 4 bytes
A double is 8 bytes
A float  is 4 bytes
A bool   is 1 bytes
\end{verbatim}

\end{framed}

\end{minipage}
\end{minipage}

\vspace{2mm}\noindent\hrulefill{}

\begin{minipage}{\linewidth}\noindent
{\tiny 042.}\\
\begin{minipage}[t]{.485\linewidth}

What is the output of:

\begin{framed}

\begin{Shaded}
\begin{Highlighting}[]
\DataTypeTok{char}\NormalTok{ a = }\DecValTok{100}\NormalTok{;}
\DataTypeTok{char}\NormalTok{ b = }\DecValTok{20}\NormalTok{;}
\DataTypeTok{char}\NormalTok{ c = a + b;}
\BuiltInTok{std::}\NormalTok{cout << }\StringTok{"a : "}\NormalTok{ << (}\DataTypeTok{int}\NormalTok{)a << }\BuiltInTok{std::}\NormalTok{endl}
\NormalTok{          << }\StringTok{"b : "}\NormalTok{ << (}\DataTypeTok{int}\NormalTok{)b << }\BuiltInTok{std::}\NormalTok{endl}
\NormalTok{          << }\StringTok{"a+b : "}\NormalTok{ << (}\DataTypeTok{int}\NormalTok{)c << }\BuiltInTok{std::}\NormalTok{endl;}
\end{Highlighting}
\end{Shaded}

\end{framed}

\end{minipage}
\hfill
\begin{minipage}[t]{.485\linewidth}

Not surprising, that is the output.

\begin{framed}

\begin{verbatim}
a : 100
b : 20
a+b : 120
\end{verbatim}

\end{framed}

\end{minipage}
\end{minipage}

\vspace{2mm}\noindent\hrulefill{}

\begin{minipage}{\linewidth}\noindent
{\tiny 043.}\\
\begin{minipage}[t]{.485\linewidth}

What is the output of:

\begin{framed}

\begin{Shaded}
\begin{Highlighting}[]
\DataTypeTok{char}\NormalTok{ a = }\DecValTok{100}\NormalTok{;}
\DataTypeTok{char}\NormalTok{ b = }\DecValTok{30}\NormalTok{;}
\DataTypeTok{char}\NormalTok{ c = a + b;}
\BuiltInTok{std::}\NormalTok{cout << }\StringTok{"a : "}\NormalTok{ << (}\DataTypeTok{int}\NormalTok{)a << }\BuiltInTok{std::}\NormalTok{endl}
\NormalTok{          << }\StringTok{"b : "}\NormalTok{ << (}\DataTypeTok{int}\NormalTok{)b << }\BuiltInTok{std::}\NormalTok{endl}
\NormalTok{          << }\StringTok{"a+b : "}\NormalTok{ << (}\DataTypeTok{int}\NormalTok{)c << }\BuiltInTok{std::}\NormalTok{endl;}
\end{Highlighting}
\end{Shaded}

\end{framed}

\end{minipage}
\hfill
\begin{minipage}[t]{.485\linewidth}

The output is:\footnote{This output may change depending on the compiler
  you are using, it could happened that you don't see anything wrong at
  all. In that case, try changing \texttt{100} for \texttt{220}.}

\begin{framed}

\begin{verbatim}
a : 100
b : 30
a+b : -126
\end{verbatim}

\end{framed}

well, that's surprising! What the heck happened?

The answer lies in the 8 bit part that I was talking about.
\VERB|\DataTypeTok{char}| can only hold 256 different numbers, but
usually we use one of those bits to indicate the sign\footnote{For more
  details look at ``two's complement binary representation''}, therefore
we have left only 7 bits for the number. \(2^7\) is \(128\), so we can
store 128 positive numbers and 128 negative numbers. If we did it
naïvely we could represent the numbers from 0 to 127 with 7 bits plus
one bit for the sign, but that's rarely used, we would be representing 0
in two ways +0 and -0, the answer is to count from 0 to 127 and from
-128 to -1, i.e., we can lay down all the representable numbers by 8
bits in the following way: -128, -127, -126, \ldots{}, -2, -1, 0, 1, 2,
\ldots{}, 125, 126, 127.

When you pass over the limit of what 7 bits\footnote{this is called
  overflow} can store you need to go somewhere, and by convention that
is going back to the first number, i.e., adding numbers one by one will
lead you to the begining no matter where you start: \(1+1=2\),
\(2+1=3\), \ldots{}, \(125+1=126\), \(127+1=-128\), \(-128+1=-127\),
\ldots{}, \(-1+1=0\), \(0+1=1\).

\end{minipage}
\end{minipage}

\vspace{2mm}\noindent\hrulefill{}

\begin{minipage}{\linewidth}\noindent
{\tiny 044.}\\
\begin{minipage}[t]{.485\linewidth}

What is the output of:

\begin{framed}

\begin{Shaded}
\begin{Highlighting}[]
\DataTypeTok{char}\NormalTok{ i = }\DecValTok{126}\NormalTok{;}
\BuiltInTok{std::}\NormalTok{cout << (}\DataTypeTok{int}\NormalTok{)i << }\StringTok{" "}\NormalTok{;}
\NormalTok{i = i + }\DecValTok{1}\NormalTok{;}
\BuiltInTok{std::}\NormalTok{cout << (}\DataTypeTok{int}\NormalTok{)i << }\StringTok{" "}\NormalTok{;}
\NormalTok{i = i + }\DecValTok{1}\NormalTok{;}
\BuiltInTok{std::}\NormalTok{cout << (}\DataTypeTok{int}\NormalTok{)i << }\StringTok{" "}\NormalTok{;}
\NormalTok{i = i + }\DecValTok{1}\NormalTok{;}
\BuiltInTok{std::}\NormalTok{cout << (}\DataTypeTok{int}\NormalTok{)i << }\BuiltInTok{std::}\NormalTok{endl;}
\end{Highlighting}
\end{Shaded}

\end{framed}

\end{minipage}
\hfill
\begin{minipage}[t]{.485\linewidth}

The output, as you may have easily guessed is:

\begin{framed}

\begin{verbatim}
126 127 -128 -127
\end{verbatim}

\end{framed}

\end{minipage}
\end{minipage}

\vspace{2mm}\noindent\hrulefill{}

\newpage

\hypertarget{interlude-how-are-numbers-represented}{%
\subsection{Interlude: How are numbers
represented?}\label{interlude-how-are-numbers-represented}}

\inlinetodo{fill me!}

\newpage

\vspace{2mm}\noindent\hrulefill{}

\begin{minipage}{\linewidth}\noindent
{\tiny 045.}\\
\begin{minipage}[t]{.485\linewidth}

What is the output of:

\begin{framed}

\begin{Shaded}
\begin{Highlighting}[]
\DataTypeTok{int}\NormalTok{ i = }\DecValTok{0}\NormalTok{;}
\ControlFlowTok{if}\NormalTok{ (i<}\DecValTok{3}\NormalTok{) \{}
  \BuiltInTok{std::}\NormalTok{cout << }\StringTok{"There is no else statement"}\NormalTok{;}
\NormalTok{  i = i * }\DecValTok{2}\NormalTok{;}
\NormalTok{\}}
\BuiltInTok{std::}\NormalTok{cout << }\BuiltInTok{std::}\NormalTok{endl;}
\end{Highlighting}
\end{Shaded}

\end{framed}

\end{minipage}
\hfill
\begin{minipage}[t]{.485\linewidth}

The output is:

\begin{framed}

\begin{verbatim}
There is no else statement
\end{verbatim}

\end{framed}

Notice how we ignored the \VERB|\ControlFlowTok{else}| statement, well
that's ok, we can just do stuff for when something is true otherwise we
don't do anything.

\end{minipage}
\end{minipage}

\vspace{2mm}\noindent\hrulefill{}

\begin{minipage}{\linewidth}\noindent
{\tiny 046.}\\
\begin{minipage}[t]{.485\linewidth}

What is the output of:

\begin{framed}

\begin{Shaded}
\begin{Highlighting}[]
\DataTypeTok{int}\NormalTok{ i = }\DecValTok{0}\NormalTok{;}
\ControlFlowTok{if}\NormalTok{ (i<}\DecValTok{3}\NormalTok{) \{}
  \BuiltInTok{std::}\NormalTok{cout << i << }\StringTok{" "}\NormalTok{;}
\NormalTok{  i = i + }\DecValTok{1}\NormalTok{;}
\NormalTok{\}}
\ControlFlowTok{if}\NormalTok{ (i<}\DecValTok{3}\NormalTok{) \{}
  \BuiltInTok{std::}\NormalTok{cout << i << }\StringTok{" "}\NormalTok{;}
\NormalTok{  i = i + }\DecValTok{1}\NormalTok{;}
\NormalTok{\}}
\BuiltInTok{std::}\NormalTok{cout << }\BuiltInTok{std::}\NormalTok{endl;}
\end{Highlighting}
\end{Shaded}

\end{framed}

\end{minipage}
\hfill
\begin{minipage}[t]{.485\linewidth}

The output is:

\begin{framed}

\begin{verbatim}
0 1 
\end{verbatim}

\end{framed}

\end{minipage}
\end{minipage}

\vspace{2mm}\noindent\hrulefill{}

\begin{minipage}{\linewidth}\noindent
{\tiny 047.}\\
\begin{minipage}[t]{.485\linewidth}

What is the output of:

\begin{framed}

\begin{Shaded}
\begin{Highlighting}[]
\DataTypeTok{int}\NormalTok{ i = }\DecValTok{0}\NormalTok{;}
\ControlFlowTok{if}\NormalTok{ (i<}\DecValTok{3}\NormalTok{) \{}
  \BuiltInTok{std::}\NormalTok{cout << i << }\StringTok{" "}\NormalTok{;}
\NormalTok{  i = i + }\DecValTok{1}\NormalTok{;}
\NormalTok{\}}
\ControlFlowTok{if}\NormalTok{ (i<}\DecValTok{3}\NormalTok{) \{}
  \BuiltInTok{std::}\NormalTok{cout << i << }\StringTok{" "}\NormalTok{;}
\NormalTok{  i = i + }\DecValTok{1}\NormalTok{;}
\NormalTok{\}}
\ControlFlowTok{if}\NormalTok{ (i<}\DecValTok{3}\NormalTok{) \{}
  \BuiltInTok{std::}\NormalTok{cout << i << }\StringTok{" "}\NormalTok{;}
\NormalTok{  i = i + }\DecValTok{1}\NormalTok{;}
\NormalTok{\}}
\ControlFlowTok{if}\NormalTok{ (i<}\DecValTok{3}\NormalTok{) \{}
  \BuiltInTok{std::}\NormalTok{cout << i << }\StringTok{" "}\NormalTok{;}
\NormalTok{  i = i + }\DecValTok{1}\NormalTok{;}
\NormalTok{\}}
\BuiltInTok{std::}\NormalTok{cout << }\BuiltInTok{std::}\NormalTok{endl;}
\end{Highlighting}
\end{Shaded}

\end{framed}

\end{minipage}
\hfill
\begin{minipage}[t]{.485\linewidth}

The output is:

\begin{framed}

\begin{verbatim}
0 1 2 
\end{verbatim}

\end{framed}

Notice how we wrote FOUR (4) \texttt{if} statements but only 3 numbers
appear. Why?

(answer, well, we start with \texttt{i==0}, then we add \texttt{1} and
print, and we do it again, and again, and again, and every time we ask
if \texttt{i} is smaller than 3, and guess what, it isn't always smaller
than 3, not forever)

\end{minipage}
\end{minipage}

\vspace{2mm}\noindent\hrulefill{}

\begin{minipage}{\linewidth}\noindent
{\tiny 048.}\\
\begin{minipage}[t]{.485\linewidth}

But what if we wanted to use a different limit, not 3 but 18, then we
would need at least 20 if statements just like the one above and that's
kinda stupid.

Introducing\ldots{} \VERB|\ControlFlowTok{while}|, it works like many
\texttt{if} one after the other with no end, it stops when the question
returns \texttt{false}. Let's see an example:

\begin{framed}

\begin{Shaded}
\begin{Highlighting}[]
\DataTypeTok{int}\NormalTok{ i = }\DecValTok{0}\NormalTok{;}
\ControlFlowTok{while}\NormalTok{ (i<}\DecValTok{18}\NormalTok{) \{}
  \BuiltInTok{std::}\NormalTok{cout << i << }\StringTok{" "}\NormalTok{;}
\NormalTok{  i = i + }\DecValTok{1}\NormalTok{;}
\NormalTok{\}}
\BuiltInTok{std::}\NormalTok{cout << }\BuiltInTok{std::}\NormalTok{endl;}
\end{Highlighting}
\end{Shaded}

\end{framed}

\end{minipage}
\hfill
\begin{minipage}[t]{.485\linewidth}

And the output as we expected, eighteen numbers:

\begin{framed}

\begin{verbatim}
0 1 2 3 4 5 6 7 8 9 10 11 12 13 14 15 16 17 
\end{verbatim}

\end{framed}

\end{minipage}
\end{minipage}

\hypertarget{block-2-more-space-to-play-with}{%
\chapter{Block 2: More space to play
with}\label{block-2-more-space-to-play-with}}

\hypertarget{block-3-deprecated-stuff-that-you-better-know-cos-everybody-uses-it}{%
\chapter{\texorpdfstring{Block 3: \textbf{deprecated} stuff that you
better know cos everybody uses
it}{Block 3: deprecated stuff that you better know cos everybody uses it}}\label{block-3-deprecated-stuff-that-you-better-know-cos-everybody-uses-it}}

\end{document}
