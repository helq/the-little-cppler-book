\PassOptionsToPackage{unicode=true}{hyperref} % options for packages loaded elsewhere
\PassOptionsToPackage{hyphens}{url}
%
\documentclass[]{article}
\usepackage{lmodern}
\usepackage{amssymb,amsmath}
\usepackage{ifxetex,ifluatex}
\usepackage{fixltx2e} % provides \textsubscript
\ifnum 0\ifxetex 1\fi\ifluatex 1\fi=0 % if pdftex
  \usepackage[T1]{fontenc}
  \usepackage[utf8]{inputenc}
  \usepackage{textcomp} % provides euro and other symbols
\else % if luatex or xelatex
  \usepackage{unicode-math}
  \defaultfontfeatures{Ligatures=TeX,Scale=MatchLowercase}
\fi
% use upquote if available, for straight quotes in verbatim environments
\IfFileExists{upquote.sty}{\usepackage{upquote}}{}
% use microtype if available
\IfFileExists{microtype.sty}{%
\usepackage[]{microtype}
\UseMicrotypeSet[protrusion]{basicmath} % disable protrusion for tt fonts
}{}
\IfFileExists{parskip.sty}{%
\usepackage{parskip}
}{% else
\setlength{\parindent}{0pt}
\setlength{\parskip}{6pt plus 2pt minus 1pt}
}
\usepackage{fancyvrb}
\usepackage{hyperref}
\hypersetup{
            pdftitle={The Little CPPler},
            pdfborder={0 0 0},
            breaklinks=true}
\urlstyle{same}  % don't use monospace font for urls
\VerbatimFootnotes % allows verbatim text in footnotes
\usepackage[verbose,a4paper,textwidth=190mm,textheight=260mm,voffset=7mm]{geometry}
\usepackage{color}
\usepackage{fancyvrb}
\newcommand{\VerbBar}{|}
\newcommand{\VERB}{\Verb[commandchars=\\\{\}]}
\DefineVerbatimEnvironment{Highlighting}{Verbatim}{commandchars=\\\{\}}
% Add ',fontsize=\small' for more characters per line
\newenvironment{Shaded}{}{}
\newcommand{\AlertTok}[1]{\textcolor[rgb]{1.00,0.00,0.00}{\textbf{#1}}}
\newcommand{\AnnotationTok}[1]{\textcolor[rgb]{0.38,0.63,0.69}{\textbf{\textit{#1}}}}
\newcommand{\AttributeTok}[1]{\textcolor[rgb]{0.49,0.56,0.16}{#1}}
\newcommand{\BaseNTok}[1]{\textcolor[rgb]{0.25,0.63,0.44}{#1}}
\newcommand{\BuiltInTok}[1]{#1}
\newcommand{\CharTok}[1]{\textcolor[rgb]{0.25,0.44,0.63}{#1}}
\newcommand{\CommentTok}[1]{\textcolor[rgb]{0.38,0.63,0.69}{\textit{#1}}}
\newcommand{\CommentVarTok}[1]{\textcolor[rgb]{0.38,0.63,0.69}{\textbf{\textit{#1}}}}
\newcommand{\ConstantTok}[1]{\textcolor[rgb]{0.53,0.00,0.00}{#1}}
\newcommand{\ControlFlowTok}[1]{\textcolor[rgb]{0.00,0.44,0.13}{\textbf{#1}}}
\newcommand{\DataTypeTok}[1]{\textcolor[rgb]{0.56,0.13,0.00}{#1}}
\newcommand{\DecValTok}[1]{\textcolor[rgb]{0.25,0.63,0.44}{#1}}
\newcommand{\DocumentationTok}[1]{\textcolor[rgb]{0.73,0.13,0.13}{\textit{#1}}}
\newcommand{\ErrorTok}[1]{\textcolor[rgb]{1.00,0.00,0.00}{\textbf{#1}}}
\newcommand{\ExtensionTok}[1]{#1}
\newcommand{\FloatTok}[1]{\textcolor[rgb]{0.25,0.63,0.44}{#1}}
\newcommand{\FunctionTok}[1]{\textcolor[rgb]{0.02,0.16,0.49}{#1}}
\newcommand{\ImportTok}[1]{#1}
\newcommand{\InformationTok}[1]{\textcolor[rgb]{0.38,0.63,0.69}{\textbf{\textit{#1}}}}
\newcommand{\KeywordTok}[1]{\textcolor[rgb]{0.00,0.44,0.13}{\textbf{#1}}}
\newcommand{\NormalTok}[1]{#1}
\newcommand{\OperatorTok}[1]{\textcolor[rgb]{0.40,0.40,0.40}{#1}}
\newcommand{\OtherTok}[1]{\textcolor[rgb]{0.00,0.44,0.13}{#1}}
\newcommand{\PreprocessorTok}[1]{\textcolor[rgb]{0.74,0.48,0.00}{#1}}
\newcommand{\RegionMarkerTok}[1]{#1}
\newcommand{\SpecialCharTok}[1]{\textcolor[rgb]{0.25,0.44,0.63}{#1}}
\newcommand{\SpecialStringTok}[1]{\textcolor[rgb]{0.73,0.40,0.53}{#1}}
\newcommand{\StringTok}[1]{\textcolor[rgb]{0.25,0.44,0.63}{#1}}
\newcommand{\VariableTok}[1]{\textcolor[rgb]{0.10,0.09,0.49}{#1}}
\newcommand{\VerbatimStringTok}[1]{\textcolor[rgb]{0.25,0.44,0.63}{#1}}
\newcommand{\WarningTok}[1]{\textcolor[rgb]{0.38,0.63,0.69}{\textbf{\textit{#1}}}}
\setlength{\emergencystretch}{3em}  % prevent overfull lines
\providecommand{\tightlist}{%
  \setlength{\itemsep}{0pt}\setlength{\parskip}{0pt}}
\setcounter{secnumdepth}{0}
% Redefines (sub)paragraphs to behave more like sections
\ifx\paragraph\undefined\else
\let\oldparagraph\paragraph
\renewcommand{\paragraph}[1]{\oldparagraph{#1}\mbox{}}
\fi
\ifx\subparagraph\undefined\else
\let\oldsubparagraph\subparagraph
\renewcommand{\subparagraph}[1]{\oldsubparagraph{#1}\mbox{}}
\fi

% set default figure placement to htbp
\makeatletter
\def\fps@figure{htbp}
\makeatother

\usepackage{framed}
%\renewenvironment{Shaded}{\begin{framed}}{\end{framed}}
%\usepackage{mdframed}
%\renewenvironment{Shaded}{\begin{mdframed}}{\end{mdframed}}

\usepackage{graphicx}

%\usepackage{layout} % package to look at the page layout

%\usepackage{fancyhdr}
%\usepackage{lastpage}
%\pagestyle{fancy}
%\lhead{hi}
%\rhead{ho}

\usepackage[l2tabu, orthodox]{nag} % this is to make sure I use some modern LaTeX tactics

%%%
% Custom footnote taken from https://tex.stackexchange.com/a/78227
\makeatletter
\newcommand*{\myfnsymbolsingle}[1]{%
  \ensuremath{%
    \ifcase#1% 0
    \or % 1
      \dagger
    \or % 2
      \ddagger
    \or % 3
      \mathsection
    \else % >= 4
      \@ctrerr
    \fi
  }%
}
\makeatother

\newcommand*{\myfnsymbol}[1]{%
  \myfnsymbolsingle{\value{#1}}%
}

% remove upper boundary by multiplying the symbols if needed
\usepackage{alphalph}
\newalphalph{\myfnsymbolmult}[mult]{\myfnsymbolsingle}{}

\renewcommand*{\thempfootnote}{%
  \myfnsymbolmult{\value{mpfootnote}}%
}

%\renewcommand{\thefootnote}{\fnsymbol{footnote}}
%%%

\hypersetup{bookmarks=true,
            pdfauthor={},
            colorlinks=true,
            citecolor=blue,
            urlcolor=blue,
            linkcolor=magenta}
\usepackage{fontspec}
\usepackage[textsize=tiny,%
            bordercolor=yellow,%
            color=yellow,
            textwidth=4cm]{todonotes}
\usepackage{setspace} % for pandoc-citeproc-preamble
%\setmainfont{Noto Serif}

\newcommand{\inlinetodo}[1]{\todo[inline,size=\normalsize]{#1}}

\title{The Little CPPler}
\date{}

\begin{document}
\maketitle

\inlinetodo{idea: at some point in the code explain why is the book written in English and
no any other language}

\hypertarget{prerequisites}{%
\section{Prerequisites}\label{prerequisites}}

\hypertarget{what-to-expect-from-this-book}{%
\subsection{What to expect from this
book}\label{what-to-expect-from-this-book}}

\inlinetodo{A phrase telling them what to do with the book}

\hypertarget{tools}{%
\subsection{Tools}\label{tools}}

\inlinetodo{Tools neccesary to follow the book (for everything to work), note: the source
code can be found in a repo, thought}

\hypertarget{using-the-tools}{%
\subsection{Using the tools}\label{using-the-tools}}

\inlinetodo{Ask to open a file with a text editor}

\inlinetodo{Extend explanation on how to compile}

\inlinetodo{ask to modify the file and see what happens}

Given the file \texttt{000-hello-world.cc}:

\begin{framed}

\begin{Shaded}
\begin{Highlighting}[]
\PreprocessorTok{#include }\ImportTok{<iostream>}

\DataTypeTok{int}\NormalTok{ main()}
\NormalTok{\{}
  \BuiltInTok{std::}\NormalTok{cout << }\StringTok{"Hello World!"}\NormalTok{ << }\BuiltInTok{std::}\NormalTok{endl;}
  \ControlFlowTok{return} \DecValTok{0}\NormalTok{;}
\NormalTok{\}}
\end{Highlighting}
\end{Shaded}

\end{framed}

compile the file into a executable with either \texttt{clang++} or
\texttt{g++}:

\inlinetodo{add std option to make it C++11 compilant}

\begin{framed}

\begin{Shaded}
\begin{Highlighting}[]
\FunctionTok{clang}\NormalTok{++ 000-hello-world.cc -o 000-hello-world.exe}
\ExtensionTok{g++}\NormalTok{ 000-hello-world.cc -o 000-hello-world.exe}
\end{Highlighting}
\end{Shaded}

\end{framed}

and run it with:

\begin{framed}

\begin{Shaded}
\begin{Highlighting}[]
\ExtensionTok{./000-hello-world.exe}
\end{Highlighting}
\end{Shaded}

\end{framed}

the output of the program in terminal will be:

\begin{framed}

\begin{verbatim}
Hello World!
\end{verbatim}

\end{framed}

\hypertarget{content}{%
\section{Content}\label{content}}

\hypertarget{how-to-read-this-book}{%
\subsection{How to read this book}\label{how-to-read-this-book}}

Yeah, you should learn first how to read this book! :P

\inlinetodo{Explain how the book is supposed to be read (try to answer question with answer
covered, then reveal answer and try understand what is it doing)}

\hypertarget{basics}{%
\subsection{Basics}\label{basics}}

\hypertarget{getting-to-know-c11}{%
\subsubsection{Getting to know C++(11)}\label{getting-to-know-c11}}

\inlinetodo{Add interludes asking whoever is reading to pause for a while to recover from
so much info}

\vspace{2mm}\noindent\hrulefill{}

\noindent
{\tiny 000.}\\
\begin{minipage}[t]{.485\linewidth}

What do you think the following code will output after compiling and
running it?

\begin{framed}

\begin{Shaded}
\begin{Highlighting}[]
\PreprocessorTok{#include }\ImportTok{<iostream>}

\DataTypeTok{int}\NormalTok{ main()}
\NormalTok{\{}
  \BuiltInTok{std::}\NormalTok{cout << }\StringTok{"Hello World!"}\NormalTok{ << }\BuiltInTok{std::}\NormalTok{endl;}
  \ControlFlowTok{return} \DecValTok{0}\NormalTok{;}
\NormalTok{\}}
\end{Highlighting}
\end{Shaded}

\end{framed}

\end{minipage}
\hfill
\begin{minipage}[t]{.485\linewidth}

The output is:

\begin{framed}

\begin{verbatim}
Hello World!
\end{verbatim}

\end{framed}

\end{minipage}

\vspace{2mm}\noindent\hrulefill{}

\noindent
{\tiny 001.}\\
\begin{minipage}[t]{.485\linewidth}

Now, what if you compile this other file:

\begin{framed}

\begin{Shaded}
\begin{Highlighting}[]
\PreprocessorTok{#include }\ImportTok{<iostream>}

\DataTypeTok{int}\NormalTok{ main()}
\NormalTok{\{}
  \BuiltInTok{std::}\NormalTok{cout << }\StringTok{"Hello World!"}\NormalTok{ << }\BuiltInTok{std::}\NormalTok{endl;}
  \BuiltInTok{std::}\NormalTok{cout << }\StringTok{"I'm a program example and I'm "}
\NormalTok{            << }\StringTok{"in English."}
\NormalTok{            << }\BuiltInTok{std::}\NormalTok{endl;}
  \ControlFlowTok{return} \DecValTok{0}\NormalTok{;}
\NormalTok{\}}
\end{Highlighting}
\end{Shaded}

\end{framed}

\end{minipage}
\hfill
\begin{minipage}[t]{.485\linewidth}

The output is:

\begin{framed}

\begin{verbatim}
Hello World!
I'm a program example and I'm in English.
\end{verbatim}

\end{framed}

Pay close attention to the output, there are three sentences surrounded
by quotation marks (\texttt{"}), but there are only two lines in the
output. Why?

\end{minipage}

\vspace{2mm}\noindent\hrulefill{}

\noindent
{\tiny 002.}\\
\begin{minipage}[t]{.485\linewidth}

If we change our example slightly (notice the semicolon (\texttt{;}))
what do you think it will happen?

\begin{framed}

\begin{Shaded}
\begin{Highlighting}[]
\PreprocessorTok{#include }\ImportTok{<iostream>}

\DataTypeTok{int}\NormalTok{ main()}
\NormalTok{\{}
  \BuiltInTok{std::}\NormalTok{cout << }\StringTok{"Hello World!"}\NormalTok{ << }\BuiltInTok{std::}\NormalTok{endl;}
  \BuiltInTok{std::}\NormalTok{cout << }\StringTok{"I'm a program example and I'm "}\NormalTok{;}
\NormalTok{            << }\StringTok{"in English."}
\NormalTok{            << }\BuiltInTok{std::}\NormalTok{endl;}
  \ControlFlowTok{return} \DecValTok{0}\NormalTok{;}
\NormalTok{\}}
\end{Highlighting}
\end{Shaded}

\end{framed}

\end{minipage}
\hfill
\begin{minipage}[t]{.485\linewidth}

Well, it doesn't compiles! We get an error similar to:

\begin{framed}

\begin{verbatim}
test.cc:7:13: error: expected expression
            << "in English."
            ^
\end{verbatim}

\end{framed}

It is telling us that it was expecting something (a \texttt{std::cout}
for example) before \texttt{\textless{}\textless{}}.

Try removing or adding random characters (anywhere) to the example and
you will find that the compiler just admits a certain arrangement of
characters and not much more. But, why? Well, the compiler just
understands the grammar of C++ as we just understand the grammar of our
human languages. Going a little further with the analogy, we can
understand the grammar of any human language (its parts (verbs,
prepositions, \ldots{}) and how are they connected) but we can only
understand the meaning (semantics) of those languages we have studied
(or our mother tongues).

\end{minipage}

\vspace{2mm}\noindent\hrulefill{}

\noindent
{\tiny 003.}\\
\begin{minipage}[t]{.485\linewidth}

Does the following program compiles. If yes, what is its output?

\begin{framed}

\begin{Shaded}
\begin{Highlighting}[]
\PreprocessorTok{#include }\ImportTok{<iostream>}

\DataTypeTok{int}\NormalTok{ main()}
\NormalTok{\{}
  \BuiltInTok{std::}\NormalTok{cout << }\StringTok{"Hello World!"}\NormalTok{ << }\BuiltInTok{std::}\NormalTok{endl;}
  \BuiltInTok{std::}\NormalTok{cout << }\StringTok{"I'm a program"}
\NormalTok{            << }\BuiltInTok{std::}\NormalTok{endl}
\NormalTok{            << }\StringTok{"example and I'm"}
\NormalTok{            << }\BuiltInTok{std::}\NormalTok{endl}
\NormalTok{            << }\StringTok{"in English."}
\NormalTok{            << }\BuiltInTok{std::}\NormalTok{endl;}
  \ControlFlowTok{return} \DecValTok{0}\NormalTok{;}
\NormalTok{\}}
\end{Highlighting}
\end{Shaded}

\end{framed}

\end{minipage}
\hfill
\begin{minipage}[t]{.485\linewidth}

Yep, it in fact compiles, and its output is:

\begin{framed}

\begin{verbatim}
Hello World!
I'm a program
example and I'm
in English.
\end{verbatim}

\end{framed}

Notice how \texttt{std::endl} puts text in a new line, that's in fact
its whole job.

\end{minipage}

\vspace{2mm}\noindent\hrulefill{}

\noindent
{\tiny 004.}\\
\begin{minipage}[t]{.485\linewidth}

Well that's getting boring. What if we try something different for a
change. What is the otput of this program:

\begin{framed}

\begin{Shaded}
\begin{Highlighting}[]
\PreprocessorTok{#include }\ImportTok{<iostream>}

\DataTypeTok{int}\NormalTok{ main()}
\NormalTok{\{}
  \BuiltInTok{std::}\NormalTok{cout << }\StringTok{"Adding two numbers: "}
\NormalTok{            << }\DecValTok{2}\NormalTok{ + }\DecValTok{3}
\NormalTok{            << }\BuiltInTok{std::}\NormalTok{endl;}
  \ControlFlowTok{return} \DecValTok{0}\NormalTok{;}
\NormalTok{\}}
\end{Highlighting}
\end{Shaded}

\end{framed}

\end{minipage}
\hfill
\begin{minipage}[t]{.485\linewidth}

Nice\footnote{this is a footnote, read all of them, they may tell you
  little things that the main text won't.}

\begin{framed}

\begin{verbatim}
Adding two numbers: 5
\end{verbatim}

\end{framed}

\end{minipage}

\vspace{2mm}\noindent\hrulefill{}

\noindent
{\tiny 005.}\\
\begin{minipage}[t]{.485\linewidth}

Let's try something a little more complex\footnote{Here you can see only
  a snippet of the whole code, the whole code representing the snippet
  is:

\begin{Shaded}
\begin{Highlighting}[]
\PreprocessorTok{#include }\ImportTok{<iostream>}

\DataTypeTok{int}\NormalTok{ main()}
\NormalTok{\{}
  \BuiltInTok{std::}\NormalTok{cout}
\NormalTok{    << }\StringTok{"A simple operation between "}\NormalTok{ << }\DecValTok{3}
\NormalTok{    << }\StringTok{" "}\NormalTok{ << }\DecValTok{5}\NormalTok{ << }\StringTok{" "}\NormalTok{ << }\DecValTok{20}\NormalTok{ << }\StringTok{": "}
\NormalTok{    << (}\DecValTok{3+5}\NormalTok{)*}\DecValTok{20}\NormalTok{ << }\BuiltInTok{std::}\NormalTok{endl;}
  \ControlFlowTok{return} \DecValTok{0}\NormalTok{;}
\NormalTok{\}}
\end{Highlighting}
\end{Shaded}

  From now on, all code will be given on snippets for simplicity but
  remember that they are that, snippets, uncomplete pieces of code that
  need your help to get complete.}

\begin{framed}

\begin{Shaded}
\begin{Highlighting}[]
\BuiltInTok{std::}\NormalTok{cout}
\NormalTok{  << }\StringTok{"A simple operation between "}\NormalTok{ << }\DecValTok{3}
\NormalTok{  << }\StringTok{" "}\NormalTok{ << }\DecValTok{5}\NormalTok{ << }\StringTok{" "}\NormalTok{ << }\DecValTok{20}\NormalTok{ << }\StringTok{": "}
\NormalTok{  << (}\DecValTok{3+5}\NormalTok{)*}\DecValTok{20}\NormalTok{ << }\BuiltInTok{std::}\NormalTok{endl;}
\end{Highlighting}
\end{Shaded}

\end{framed}

\end{minipage}
\hfill
\begin{minipage}[t]{.485\linewidth}

\begin{framed}

\begin{verbatim}
A simple operation between 3 5 20: 160
\end{verbatim}

\end{framed}

\end{minipage}

\vspace{2mm}\noindent\hrulefill{}

\end{document}
