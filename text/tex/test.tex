\PassOptionsToPackage{unicode=true}{hyperref} % options for packages loaded elsewhere
\PassOptionsToPackage{hyphens}{url}
%
\documentclass[]{article}
\usepackage{lmodern}
\usepackage{amssymb,amsmath}
\usepackage{ifxetex,ifluatex}
\usepackage{fixltx2e} % provides \textsubscript
\ifnum 0\ifxetex 1\fi\ifluatex 1\fi=0 % if pdftex
  \usepackage[T1]{fontenc}
  \usepackage[utf8]{inputenc}
  \usepackage{textcomp} % provides euro and other symbols
\else % if luatex or xelatex
  \usepackage{unicode-math}
  \defaultfontfeatures{Ligatures=TeX,Scale=MatchLowercase}
\fi
% use upquote if available, for straight quotes in verbatim environments
\IfFileExists{upquote.sty}{\usepackage{upquote}}{}
% use microtype if available
\IfFileExists{microtype.sty}{%
\usepackage[]{microtype}
\UseMicrotypeSet[protrusion]{basicmath} % disable protrusion for tt fonts
}{}
\IfFileExists{parskip.sty}{%
\usepackage{parskip}
}{% else
\setlength{\parindent}{0pt}
\setlength{\parskip}{6pt plus 2pt minus 1pt}
}
\usepackage{hyperref}
\hypersetup{
            pdftitle={The Little CPPler},
            pdfborder={0 0 0},
            breaklinks=true}
\urlstyle{same}  % don't use monospace font for urls
\usepackage[verbose,a4paper,textwidth=190mm,textheight=260mm,voffset=7mm]{geometry}
\usepackage{color}
\usepackage{fancyvrb}
\newcommand{\VerbBar}{|}
\newcommand{\VERB}{\Verb[commandchars=\\\{\}]}
\DefineVerbatimEnvironment{Highlighting}{Verbatim}{commandchars=\\\{\}}
% Add ',fontsize=\small' for more characters per line
\newenvironment{Shaded}{}{}
\newcommand{\AlertTok}[1]{\textcolor[rgb]{1.00,0.00,0.00}{\textbf{#1}}}
\newcommand{\AnnotationTok}[1]{\textcolor[rgb]{0.38,0.63,0.69}{\textbf{\textit{#1}}}}
\newcommand{\AttributeTok}[1]{\textcolor[rgb]{0.49,0.56,0.16}{#1}}
\newcommand{\BaseNTok}[1]{\textcolor[rgb]{0.25,0.63,0.44}{#1}}
\newcommand{\BuiltInTok}[1]{#1}
\newcommand{\CharTok}[1]{\textcolor[rgb]{0.25,0.44,0.63}{#1}}
\newcommand{\CommentTok}[1]{\textcolor[rgb]{0.38,0.63,0.69}{\textit{#1}}}
\newcommand{\CommentVarTok}[1]{\textcolor[rgb]{0.38,0.63,0.69}{\textbf{\textit{#1}}}}
\newcommand{\ConstantTok}[1]{\textcolor[rgb]{0.53,0.00,0.00}{#1}}
\newcommand{\ControlFlowTok}[1]{\textcolor[rgb]{0.00,0.44,0.13}{\textbf{#1}}}
\newcommand{\DataTypeTok}[1]{\textcolor[rgb]{0.56,0.13,0.00}{#1}}
\newcommand{\DecValTok}[1]{\textcolor[rgb]{0.25,0.63,0.44}{#1}}
\newcommand{\DocumentationTok}[1]{\textcolor[rgb]{0.73,0.13,0.13}{\textit{#1}}}
\newcommand{\ErrorTok}[1]{\textcolor[rgb]{1.00,0.00,0.00}{\textbf{#1}}}
\newcommand{\ExtensionTok}[1]{#1}
\newcommand{\FloatTok}[1]{\textcolor[rgb]{0.25,0.63,0.44}{#1}}
\newcommand{\FunctionTok}[1]{\textcolor[rgb]{0.02,0.16,0.49}{#1}}
\newcommand{\ImportTok}[1]{#1}
\newcommand{\InformationTok}[1]{\textcolor[rgb]{0.38,0.63,0.69}{\textbf{\textit{#1}}}}
\newcommand{\KeywordTok}[1]{\textcolor[rgb]{0.00,0.44,0.13}{\textbf{#1}}}
\newcommand{\NormalTok}[1]{#1}
\newcommand{\OperatorTok}[1]{\textcolor[rgb]{0.40,0.40,0.40}{#1}}
\newcommand{\OtherTok}[1]{\textcolor[rgb]{0.00,0.44,0.13}{#1}}
\newcommand{\PreprocessorTok}[1]{\textcolor[rgb]{0.74,0.48,0.00}{#1}}
\newcommand{\RegionMarkerTok}[1]{#1}
\newcommand{\SpecialCharTok}[1]{\textcolor[rgb]{0.25,0.44,0.63}{#1}}
\newcommand{\SpecialStringTok}[1]{\textcolor[rgb]{0.73,0.40,0.53}{#1}}
\newcommand{\StringTok}[1]{\textcolor[rgb]{0.25,0.44,0.63}{#1}}
\newcommand{\VariableTok}[1]{\textcolor[rgb]{0.10,0.09,0.49}{#1}}
\newcommand{\VerbatimStringTok}[1]{\textcolor[rgb]{0.25,0.44,0.63}{#1}}
\newcommand{\WarningTok}[1]{\textcolor[rgb]{0.38,0.63,0.69}{\textbf{\textit{#1}}}}
\setlength{\emergencystretch}{3em}  % prevent overfull lines
\providecommand{\tightlist}{%
  \setlength{\itemsep}{0pt}\setlength{\parskip}{0pt}}
\setcounter{secnumdepth}{0}
% Redefines (sub)paragraphs to behave more like sections
\ifx\paragraph\undefined\else
\let\oldparagraph\paragraph
\renewcommand{\paragraph}[1]{\oldparagraph{#1}\mbox{}}
\fi
\ifx\subparagraph\undefined\else
\let\oldsubparagraph\subparagraph
\renewcommand{\subparagraph}[1]{\oldsubparagraph{#1}\mbox{}}
\fi

% set default figure placement to htbp
\makeatletter
\def\fps@figure{htbp}
\makeatother

\usepackage{framed}
%\renewenvironment{Shaded}{\begin{framed}}{\end{framed}}
%\usepackage{mdframed}
%\renewenvironment{Shaded}{\begin{mdframed}}{\end{mdframed}}

\usepackage{graphicx}

%\usepackage{layout} % package to look at the page layout

%\usepackage{fancyhdr}
%\usepackage{lastpage}
%\pagestyle{fancy}
%\lhead{hi}
%\rhead{ho}

\usepackage[l2tabu, orthodox]{nag} % this is to make sure I use some modern LaTeX tactics

%%%
% Custom footnote taken from https://tex.stackexchange.com/a/78227
\makeatletter
\newcommand*{\myfnsymbolsingle}[1]{%
  \ensuremath{%
    \ifcase#1% 0
    \or % 1
      \dagger
    \or % 2
      \ddagger
    \or % 3
      \mathsection
    \else % >= 4
      \@ctrerr
    \fi
  }%
}
\makeatother

\newcommand*{\myfnsymbol}[1]{%
  \myfnsymbolsingle{\value{#1}}%
}

% remove upper boundary by multiplying the symbols if needed
\usepackage{alphalph}
\newalphalph{\myfnsymbolmult}[mult]{\myfnsymbolsingle}{}

\renewcommand*{\thempfootnote}{%
  \myfnsymbolmult{\value{mpfootnote}}%
}

%\renewcommand{\thefootnote}{\fnsymbol{footnote}}
%%%

\hypersetup{bookmarks=true,
            pdfauthor={},
            colorlinks=true,
            citecolor=blue,
            urlcolor=blue,
            linkcolor=magenta}
\usepackage{fontspec}
\usepackage[textsize=tiny,%
            bordercolor=yellow,%
            color=yellow,
            textwidth=4cm]{todonotes}
\usepackage{setspace} % for pandoc-citeproc-preamble
%\setmainfont{Noto Serif}

\newcommand{\inlinetodo}[1]{\todo[inline,size=\normalsize]{#1}}

\title{The Little CPPler}
\date{}

\begin{document}
\maketitle

\inlinetodo{idea: at some point in the code explain why is the book written in English and
no any other language}

\hypertarget{prerequisites}{%
\section{Prerequisites}\label{prerequisites}}

\hypertarget{what-to-expect-from-this-book}{%
\subsection{What to expect from this
book}\label{what-to-expect-from-this-book}}

\inlinetodo{A phrase telling them what to do with the book}

\hypertarget{tools}{%
\subsection{Tools}\label{tools}}

\inlinetodo{Tools neccesary to follow the book (for everything to work), note: the source
code can be found in a repo, though}

\hypertarget{using-the-tools}{%
\subsection{Using the tools}\label{using-the-tools}}

\inlinetodo{Ask to open a file with a text editor}

\inlinetodo{Extend explanation on how to compile}

\inlinetodo{ask to modify the file and see what happens}

Given the file \texttt{000-hello-world.cc}:

\begin{framed}

\begin{Shaded}
\begin{Highlighting}[]
\PreprocessorTok{#include }\ImportTok{<iostream>}

\DataTypeTok{int}\NormalTok{ main()}
\NormalTok{\{}
  \BuiltInTok{std::}\NormalTok{cout << }\StringTok{"Hello World!"}\NormalTok{ << }\BuiltInTok{std::}\NormalTok{endl;}
  \ControlFlowTok{return} \DecValTok{0}\NormalTok{;}
\NormalTok{\}}
\end{Highlighting}
\end{Shaded}

\end{framed}

compile the file into a executable with either \texttt{clang++} or
\texttt{g++}:

\inlinetodo{add std option to make it C++11 compilant}

\begin{framed}

\begin{Shaded}
\begin{Highlighting}[]
\FunctionTok{clang}\NormalTok{++ 000-hello-world.cc -o 000-hello-world.exe}
\ExtensionTok{g++}\NormalTok{ 000-hello-world.cc -o 000-hello-world.exe}
\end{Highlighting}
\end{Shaded}

\end{framed}

and run it with:

\begin{framed}

\begin{Shaded}
\begin{Highlighting}[]
\ExtensionTok{./000-hello-world.exe}
\end{Highlighting}
\end{Shaded}

\end{framed}

the output of the program in terminal will be:

\begin{framed}

\begin{verbatim}
Hello World!
\end{verbatim}

\end{framed}

\newpage

\hypertarget{content}{%
\section{Content}\label{content}}

\hypertarget{how-to-read-this-book}{%
\subsection{How to read this book}\label{how-to-read-this-book}}

Yeah, you should learn first how to read this book! :P

\inlinetodo{Explain how the book is supposed to be read (try to answer question with answer
covered, then reveal answer and try understand what is it doing)}

\newpage

\hypertarget{basics}{%
\subsection{Basics}\label{basics}}

\hypertarget{getting-to-know-c11}{%
\subsubsection{Getting to know C++(11)}\label{getting-to-know-c11}}

\inlinetodo{Add interludes asking whoever is reading to pause for a while to recover from
so much info}

\vspace{2mm}\noindent\hrulefill{}

\begin{minipage}{\linewidth}\noindent
{\tiny 000.}\\
\begin{minipage}[t]{.485\linewidth}

What do you think the following code will output after compiling and
running it?

\begin{framed}

\begin{Shaded}
\begin{Highlighting}[]
\PreprocessorTok{#include }\ImportTok{<iostream>}

\DataTypeTok{int}\NormalTok{ main()}
\NormalTok{\{}
  \BuiltInTok{std::}\NormalTok{cout << }\StringTok{"Hello World!"}\NormalTok{ << }\BuiltInTok{std::}\NormalTok{endl;}
  \ControlFlowTok{return} \DecValTok{0}\NormalTok{;}
\NormalTok{\}}
\end{Highlighting}
\end{Shaded}

\end{framed}

\end{minipage}
\hfill
\begin{minipage}[t]{.485\linewidth}

The output is:

\begin{framed}

\begin{verbatim}
Hello World!
\end{verbatim}

\end{framed}

\end{minipage}
\end{minipage}

\vspace{2mm}\noindent\hrulefill{}

\begin{minipage}{\linewidth}\noindent
{\tiny 001.}\\
\begin{minipage}[t]{.485\linewidth}

Now, what if you compile this other file:

\begin{framed}

\begin{Shaded}
\begin{Highlighting}[]
\PreprocessorTok{#include }\ImportTok{<iostream>}

\DataTypeTok{int}\NormalTok{ main()}
\NormalTok{\{}
  \BuiltInTok{std::}\NormalTok{cout << }\StringTok{"Hello World!"}\NormalTok{ << }\BuiltInTok{std::}\NormalTok{endl;}
  \BuiltInTok{std::}\NormalTok{cout << }\StringTok{"I'm a program example and I'm "}
\NormalTok{            << }\StringTok{"in English."}
\NormalTok{            << }\BuiltInTok{std::}\NormalTok{endl;}
  \ControlFlowTok{return} \DecValTok{0}\NormalTok{;}
\NormalTok{\}}
\end{Highlighting}
\end{Shaded}

\end{framed}

\end{minipage}
\hfill
\begin{minipage}[t]{.485\linewidth}

The output is:

\begin{framed}

\begin{verbatim}
Hello World!
I'm a program example and I'm in English.
\end{verbatim}

\end{framed}

Pay close attention to the output, there are three sentences surrounded
by quotation marks (\texttt{"}), but there are only two lines in the
output. Why?

\end{minipage}
\end{minipage}

\vspace{2mm}\noindent\hrulefill{}

\begin{minipage}{\linewidth}\noindent
{\tiny 002.}\\
\begin{minipage}[t]{.485\linewidth}

If we change our example slightly (notice the semicolon (\texttt{;}))
what do you think it will happen?

\begin{framed}

\begin{Shaded}
\begin{Highlighting}[]
\PreprocessorTok{#include }\ImportTok{<iostream>}

\DataTypeTok{int}\NormalTok{ main()}
\NormalTok{\{}
  \BuiltInTok{std::}\NormalTok{cout << }\StringTok{"Hello World!"}\NormalTok{ << }\BuiltInTok{std::}\NormalTok{endl;}
  \BuiltInTok{std::}\NormalTok{cout << }\StringTok{"I'm a program example and I'm "}\NormalTok{;}
\NormalTok{            << }\StringTok{"in English."}
\NormalTok{            << }\BuiltInTok{std::}\NormalTok{endl;}
  \ControlFlowTok{return} \DecValTok{0}\NormalTok{;}
\NormalTok{\}}
\end{Highlighting}
\end{Shaded}

\end{framed}

\end{minipage}
\hfill
\begin{minipage}[t]{.485\linewidth}

Well, it doesn't compiles! We get an error similar to:

\begin{framed}

\begin{verbatim}
test.cc:7:13: error: expected expression
            << "in English."
            ^
\end{verbatim}

\end{framed}

It is telling us that it was expecting something (a
\VERB|\BuiltInTok{std::}\NormalTok{cout}| for example) before
\VERB|\NormalTok{<<}|.

Try removing or adding random characters (anywhere) to the example and
you will find that the compiler just admits a certain arrangement of
characters and not much more. But, why? Well, the compiler just
understands the grammar of C++ as we just understand the grammar of our
human languages. Going a little further with the analogy, we can
understand the grammar of any human language (its parts (verbs,
prepositions, \ldots{}) and how are they connected) but we can only
understand the meaning (semantics) of those languages we have studied
(or our mother tongues).

\end{minipage}
\end{minipage}

\vspace{2mm}\noindent\hrulefill{}

\begin{minipage}{\linewidth}\noindent
{\tiny 003.}\\
\begin{minipage}[t]{.485\linewidth}

Does the following program compiles. If yes, what is its output?

\begin{framed}

\begin{Shaded}
\begin{Highlighting}[]
\PreprocessorTok{#include }\ImportTok{<iostream>}

\DataTypeTok{int}\NormalTok{ main()}
\NormalTok{\{}
  \BuiltInTok{std::}\NormalTok{cout << }\StringTok{"Hello World!"}\NormalTok{ << }\BuiltInTok{std::}\NormalTok{endl;}
  \BuiltInTok{std::}\NormalTok{cout << }\StringTok{"I'm a program"}
\NormalTok{            << }\BuiltInTok{std::}\NormalTok{endl}
\NormalTok{            << }\StringTok{"example and I'm"}
\NormalTok{            << }\BuiltInTok{std::}\NormalTok{endl}
\NormalTok{            << }\StringTok{"in English."}
\NormalTok{            << }\BuiltInTok{std::}\NormalTok{endl;}
  \ControlFlowTok{return} \DecValTok{0}\NormalTok{;}
\NormalTok{\}}
\end{Highlighting}
\end{Shaded}

\end{framed}

\end{minipage}
\hfill
\begin{minipage}[t]{.485\linewidth}

Yep, it in fact compiles, and its output is:

\begin{framed}

\begin{verbatim}
Hello World!
I'm a program
example and I'm
in English.
\end{verbatim}

\end{framed}

Notice how \VERB|\BuiltInTok{std::}\NormalTok{endl}| puts text in a new
line, that's in fact its whole job.

\end{minipage}
\end{minipage}

\vspace{2mm}\noindent\hrulefill{}

\begin{minipage}{\linewidth}\noindent
{\tiny 004.}\\
\begin{minipage}[t]{.485\linewidth}

Well that's getting boring. What if we try something different for a
change. What is the otput of this program:

\begin{framed}

\begin{Shaded}
\begin{Highlighting}[]
\PreprocessorTok{#include }\ImportTok{<iostream>}

\DataTypeTok{int}\NormalTok{ main()}
\NormalTok{\{}
  \BuiltInTok{std::}\NormalTok{cout << }\StringTok{"Adding two numbers: "}
\NormalTok{            << }\DecValTok{2}\NormalTok{ + }\DecValTok{3}
\NormalTok{            << }\BuiltInTok{std::}\NormalTok{endl;}
  \ControlFlowTok{return} \DecValTok{0}\NormalTok{;}
\NormalTok{\}}
\end{Highlighting}
\end{Shaded}

\end{framed}

\end{minipage}
\hfill
\begin{minipage}[t]{.485\linewidth}

Nice\footnote{this is a footnote, read all of them, they may tell you
  little things that the main text won't.}

\begin{framed}

\begin{verbatim}
Adding two numbers: 5
\end{verbatim}

\end{framed}

\end{minipage}
\end{minipage}

\vspace{2mm}\noindent\hrulefill{}

\begin{minipage}{\linewidth}\noindent
{\tiny 005.}\\
\begin{minipage}[t]{.485\linewidth}

Let's try something a little more complex\footnote{Here you can see only
  a snippet of the whole code. The complete code the snippet represents
  can be found in the source accompaning this book.

  From now on, all code will be given on snippets for simplicity but
  remember that they are that, snippets, uncomplete pieces of code that
  need your help to get complete.}

\begin{framed}

\begin{Shaded}
\begin{Highlighting}[]
\BuiltInTok{std::}\NormalTok{cout}
\NormalTok{  << }\StringTok{"A simple operation between "}\NormalTok{ << }\DecValTok{3}
\NormalTok{  << }\StringTok{" "}\NormalTok{ << }\DecValTok{5}\NormalTok{ << }\StringTok{" "}\NormalTok{ << }\DecValTok{20}\NormalTok{ << }\StringTok{": "}
\NormalTok{  << (}\DecValTok{3+5}\NormalTok{)*}\DecValTok{20}\NormalTok{ << }\BuiltInTok{std::}\NormalTok{endl;}
\end{Highlighting}
\end{Shaded}

\end{framed}

\end{minipage}
\hfill
\begin{minipage}[t]{.485\linewidth}

\begin{framed}

\begin{verbatim}
A simple operation between 3 5 20: 160
\end{verbatim}

\end{framed}

\end{minipage}
\end{minipage}

\vspace{2mm}\noindent\hrulefill{}

\begin{minipage}{\linewidth}\noindent
{\tiny 006.}\\
\begin{minipage}[t]{.485\linewidth}

What is the purpose of
\VERB|\NormalTok{<< }\StringTok{" "}\NormalTok{ <<}| in teh code?

\end{minipage}
\hfill
\begin{minipage}[t]{.485\linewidth}

\VERB|\NormalTok{<< }\StringTok{" "}\NormalTok{ <<}| adds an space
between the numbers otherwise the output would look weird.

\end{minipage}
\end{minipage}

\vspace{2mm}\noindent\hrulefill{}

\begin{minipage}{\linewidth}\noindent
{\tiny 007.}\\
\begin{minipage}[t]{.485\linewidth}

What if we remove all spaces from the last example?

\begin{framed}

\begin{Shaded}
\begin{Highlighting}[]
\BuiltInTok{std::}\NormalTok{cout}
\NormalTok{  << }\StringTok{"A simple operation between "}\NormalTok{ << }\DecValTok{3}
\NormalTok{  << }\DecValTok{5}\NormalTok{ << }\DecValTok{20}\NormalTok{ << }\StringTok{": "}
\NormalTok{  << (}\DecValTok{3+5}\NormalTok{)*}\DecValTok{20}\NormalTok{ << }\BuiltInTok{std::}\NormalTok{endl;}
\end{Highlighting}
\end{Shaded}

\end{framed}

\end{minipage}
\hfill
\begin{minipage}[t]{.485\linewidth}

\begin{framed}

\begin{verbatim}
A simple operation between 3520: 160
\end{verbatim}

\end{framed}

It looks aweful, doesn't it? Spaces are important as formatting!

\end{minipage}
\end{minipage}

\vspace{2mm}\noindent\hrulefill{}

\begin{minipage}{\linewidth}\noindent
{\tiny 008.}\\
\begin{minipage}[t]{.485\linewidth}

Let's try it now multiline:

\begin{framed}

\begin{Shaded}
\begin{Highlighting}[]
\BuiltInTok{std::}\NormalTok{cout}
\NormalTok{  << }\StringTok{"A simple operation between "}\NormalTok{ << }\BuiltInTok{std::}\NormalTok{endl}
\NormalTok{  << }\DecValTok{3}\NormalTok{ << }\BuiltInTok{std::}\NormalTok{endl}
\NormalTok{  << }\DecValTok{5}\NormalTok{ << }\BuiltInTok{std::}\NormalTok{endl}
\NormalTok{  << }\DecValTok{20}\NormalTok{ << }\StringTok{": "}\NormalTok{ << }\BuiltInTok{std::}\NormalTok{endl}
\NormalTok{  << (}\DecValTok{3+5}\NormalTok{)*}\DecValTok{20}\NormalTok{ << }\BuiltInTok{std::}\NormalTok{endl;}
\end{Highlighting}
\end{Shaded}

\end{framed}

\end{minipage}
\hfill
\begin{minipage}[t]{.485\linewidth}

\begin{framed}

\begin{verbatim}
A simple operation between 
3
5
20: 
160
\end{verbatim}

\end{framed}

\end{minipage}
\end{minipage}

\vspace{2mm}\noindent\hrulefill{}

\begin{minipage}{\linewidth}\noindent
{\tiny 009.}\\
\begin{minipage}[t]{.485\linewidth}

Did you noticed that we only used a
\VERB|\BuiltInTok{std::}\NormalTok{cout}|?

What is then the output of the code below?

\begin{framed}

\begin{Shaded}
\begin{Highlighting}[]
\BuiltInTok{std::}\NormalTok{cout}
\NormalTok{  << }\StringTok{"A simple operation between "}\NormalTok{ << }\BuiltInTok{std::}\NormalTok{endl;}
\BuiltInTok{std::}\NormalTok{cout << }\DecValTok{3}\NormalTok{ << }\BuiltInTok{std::}\NormalTok{endl;}
\BuiltInTok{std::}\NormalTok{cout << }\DecValTok{5}\NormalTok{ << }\BuiltInTok{std::}\NormalTok{endl;}
\BuiltInTok{std::}\NormalTok{cout << }\DecValTok{20}\NormalTok{ << }\StringTok{": "}\NormalTok{ << }\BuiltInTok{std::}\NormalTok{endl;}
\BuiltInTok{std::}\NormalTok{cout << (}\DecValTok{3+5}\NormalTok{)*}\DecValTok{20}\NormalTok{ << }\BuiltInTok{std::}\NormalTok{endl;}
\end{Highlighting}
\end{Shaded}

\end{framed}

\end{minipage}
\hfill
\begin{minipage}[t]{.485\linewidth}

\begin{framed}

\begin{verbatim}
A simple operation between 
3
5
20: 
160
\end{verbatim}

\end{framed}

Yeah, it's the same as before!

Notice how the semicolon (\texttt{;}) indicates the ending of a
statement in code. The code above could all be written in a single line
(and not in 6 lines) and it would output the same:\footnote{sorry, the
  single line is too long to show in here all at once.}

\begin{framed}

\begin{Shaded}
\begin{Highlighting}[]
\BuiltInTok{std::}\NormalTok{cout << ... << (}\DecValTok{3+5}\NormalTok{)*}\DecValTok{20}\NormalTok{ << }\BuiltInTok{std::}\NormalTok{endl;}
\end{Highlighting}
\end{Shaded}

\end{framed}

Remember every \VERB|\BuiltInTok{std::}\NormalTok{cout}| is always
paired with a semicolon (\texttt{;}) which indicates the ending of its
effects, like a \emph{dot} indicates the ending of a sentence or
paragraph.

\end{minipage}
\end{minipage}

\vspace{2mm}\noindent\hrulefill{}

\begin{minipage}{\linewidth}\noindent
{\tiny 010.}\\
\begin{minipage}[t]{.485\linewidth}

What happens if you try to compile and run this faulty code?

\begin{framed}

\begin{Shaded}
\begin{Highlighting}[]
\BuiltInTok{std::}\NormalTok{cout}
\NormalTok{  << }\StringTok{"A simple operation between "}\NormalTok{ << }\BuiltInTok{std::}\NormalTok{endl;}
\BuiltInTok{std::}\NormalTok{cout << }\DecValTok{3}\NormalTok{ << }\BuiltInTok{std::}\NormalTok{endl;}
\BuiltInTok{std::}\NormalTok{cout << }\DecValTok{5}\NormalTok{ << }\BuiltInTok{std::}\NormalTok{endl}
\BuiltInTok{std::}\NormalTok{cout << }\DecValTok{20}\NormalTok{ << }\StringTok{": "}\NormalTok{ << }\BuiltInTok{std::}\NormalTok{endl;}
\BuiltInTok{std::}\NormalTok{cout << (}\DecValTok{3+5}\NormalTok{)*}\DecValTok{20}\NormalTok{ << }\BuiltInTok{std::}\NormalTok{endl;}
\end{Highlighting}
\end{Shaded}

\end{framed}

\end{minipage}
\hfill
\begin{minipage}[t]{.485\linewidth}

It fails to compile because there is a semicolo missing in the code!

The error shown by the compiler is actually\footnote{Why ``actually''?
  Well, you will find that most of the time the errors thrown by the
  compiler are hard to understand, and it is often something that we
  programmers need to learn to do. We learn to understand the confusing
  error messages the compilers give us.} useful here, it is telling us
that we forgot a \texttt{;}!

\begin{framed}

\begin{verbatim}
test.cc:8:30: error: expected ';' after expression
  std::cout << 5 << std::endl
                             ^
                             ;
1 error generated.
\end{verbatim}

\end{framed}

\end{minipage}
\end{minipage}

\vspace{2mm}\noindent\hrulefill{}

\begin{minipage}{\linewidth}\noindent
{\tiny 011.}\\
\begin{minipage}[t]{.485\linewidth}

But what if we want to not input 5 or 20 twice?

What is the output of the code below?

\inlinetodo{remember to explain how to initialize using \texttt{\{\}}}

\begin{framed}

\begin{Shaded}
\begin{Highlighting}[]
\DataTypeTok{int}\NormalTok{ num_1 = }\DecValTok{3}\NormalTok{;}
\DataTypeTok{int}\NormalTok{ num_2 = }\DecValTok{5}\NormalTok{;}
\DataTypeTok{int}\NormalTok{ num_3 = }\DecValTok{20}\NormalTok{;}

\BuiltInTok{std::}\NormalTok{cout}
\NormalTok{  << }\StringTok{"A simple operation between "}
\NormalTok{  << num_1 << }\StringTok{" "}
\NormalTok{  << num_2 << }\StringTok{" "}
\NormalTok{  << num_3 << }\StringTok{": "}
\NormalTok{  << (num_1+num_2)*nu}\VariableTok{m_3}
\NormalTok{  << }\BuiltInTok{std::}\NormalTok{endl;}
\end{Highlighting}
\end{Shaded}

\end{framed}

\end{minipage}
\hfill
\begin{minipage}[t]{.485\linewidth}

The output is:

\begin{framed}

\begin{verbatim}
A simple operation between 3 5 20: 160
\end{verbatim}

\end{framed}

\end{minipage}
\end{minipage}

\vspace{2mm}\noindent\hrulefill{}

\begin{minipage}{\linewidth}\noindent
{\tiny 012.}\\
\begin{minipage}[t]{.485\linewidth}

what is the output if you change the value 5 for 7?

\end{minipage}
\hfill
\begin{minipage}[t]{.485\linewidth}

The output is:

\begin{framed}

\begin{verbatim}
A simple operation between 3 5 20: 160
\end{verbatim}

\end{framed}

\end{minipage}
\end{minipage}

\vspace{2mm}\noindent\hrulefill{}

\begin{minipage}{\linewidth}\noindent
{\tiny 013.}\\
\begin{minipage}[t]{.485\linewidth}

\texttt{num\_1} is a \textbf{variable} and it allows us to save integers
on it, you can try changing it's value for any number between
\(-2147483649\) and \(2147483648\)\footnote{This numbers are based on a
  program compiled for a 32bit computer, the real values dependent
  between different computers.}

What if we put \(-12\) in the variable \texttt{num\_1}?

\end{minipage}
\hfill
\begin{minipage}[t]{.485\linewidth}

The output is:

\begin{framed}

\begin{verbatim}
A simple operation between -12 5 20: -140
\end{verbatim}

\end{framed}

\end{minipage}
\end{minipage}

\vspace{2mm}\noindent\hrulefill{}

\newpage

\hypertarget{interlude-variables}{%
\subsubsection{Interlude: Variables}\label{interlude-variables}}

Now, it's time to explain what are \textbf{variables} and what happens
when we write \VERB|\DataTypeTok{int}\NormalTok{ name = }\DecValTok{0}|.

\VERB|\DataTypeTok{int}\NormalTok{ name = }\DecValTok{0}| is equivalent
to:\footnote{only for the simplest values \texttt{int}, \texttt{double},
  \ldots{}, but not for objects. Objects are out of the scope of this
  book, but it is important to know they exist.}

\begin{framed}

\begin{Shaded}
\begin{Highlighting}[]
\DataTypeTok{int}\NormalTok{ name;}
\NormalTok{name = }\DecValTok{0}\NormalTok{;}
\end{Highlighting}
\end{Shaded}

\end{framed}

The first instruction \textbf{declares} a space for an \texttt{int} in
memory (RAM memomry).

So, let's study the architecture of computers, mainly RAM and CPU.

\inlinetodo{ADD explanation on the architecture of computers}

\VERB|\DataTypeTok{int}\NormalTok{ name;}| then is telling the compiler
to reserve (\emph{declare}) some space that nobody else should use. This
space can have any value we want. Because this space in memory could
have been used by anybody else in the past, its value is
\emph{nondefined}, meaning that it can be anything. Therefore we use the
next line \VERB|\NormalTok{name = }\DecValTok{0}\NormalTok{;}| to save a
zero in the space declared.

BEWARE! \VERB|\NormalTok{name = }\DecValTok{0}\NormalTok{;}| is NOT an
equation!

I repeat, \VERB|\NormalTok{name = }\DecValTok{0}\NormalTok{;}| is NOT an
equation!, you are \textbf{assigning} a value to a variable, you could
easily assign many different values to a variable, though just the last
one will stay in memory:

\begin{framed}

\begin{Shaded}
\begin{Highlighting}[]
\DataTypeTok{int}\NormalTok{ name;}
\NormalTok{name = }\DecValTok{0}\NormalTok{;}
\NormalTok{name = }\DecValTok{12}\NormalTok{;}
\end{Highlighting}
\end{Shaded}

\end{framed}

The procedure of \emph{declaring} and then \emph{assigning} a value to a
variable is so common that the designers of the language have made a
shortcut:

\begin{framed}

\begin{Shaded}
\begin{Highlighting}[]
\DataTypeTok{int}\NormalTok{ name = }\DecValTok{0}\NormalTok{;}
\end{Highlighting}
\end{Shaded}

\end{framed}

Now, let's go back to the code!

\newpage

\vspace{2mm}\noindent\hrulefill{}

\begin{minipage}{\linewidth}\noindent
{\tiny 014.}\\
\begin{minipage}[t]{.485\linewidth}

What is the output of:

\begin{framed}

\begin{Shaded}
\begin{Highlighting}[]
\DataTypeTok{int}\NormalTok{ var1 = }\DecValTok{6}\NormalTok{;}
\DataTypeTok{int}\NormalTok{ var2 = }\DecValTok{3}\NormalTok{;}
\DataTypeTok{int}\NormalTok{ var3 = }\DecValTok{10}\NormalTok{;}

\BuiltInTok{std::}\NormalTok{cout}
\NormalTok{  << (var1+var2) * var3 - var}\DecValTok{2}
\NormalTok{  << }\BuiltInTok{std::}\NormalTok{endl;}
\end{Highlighting}
\end{Shaded}

\end{framed}

\end{minipage}
\hfill
\begin{minipage}[t]{.485\linewidth}

The output is:

\begin{framed}

\begin{verbatim}
87
\end{verbatim}

\end{framed}

\end{minipage}
\end{minipage}

\vspace{2mm}\noindent\hrulefill{}

\begin{minipage}{\linewidth}\noindent
{\tiny 015.}\\
\begin{minipage}[t]{.485\linewidth}

What is the output of:

\begin{framed}

\begin{Shaded}
\begin{Highlighting}[]
\DataTypeTok{int}\NormalTok{ var1 = }\DecValTok{6}\NormalTok{;}
\DataTypeTok{int}\NormalTok{ var2 = }\DecValTok{3}\NormalTok{;}

\BuiltInTok{std::}\NormalTok{cout}
\NormalTok{  << (var1+var2) * var3 - var}\DecValTok{2}
\NormalTok{  << }\BuiltInTok{std::}\NormalTok{endl;}

\DataTypeTok{int}\NormalTok{ var3 = }\DecValTok{10}\NormalTok{;}
\end{Highlighting}
\end{Shaded}

\end{framed}

\end{minipage}
\hfill
\begin{minipage}[t]{.485\linewidth}

Yeah, it doesn't compile, you are trying to use a variable \emph{before}
declaring it (asking for a space in memory to use it). The compiler
gives you the answer:

\begin{framed}

\begin{verbatim}
test.cc:9:20: error: use of undeclared identifier 'var3'
  << (var1+var2) * var3 - var2
                   ^
1 error generated.
\end{verbatim}

\end{framed}

ORDER (of sentences) is key! It is not the same to say ``Peter eats
spaguetti, then Peter clean his teeth'' than ``Peter clean his teeth,
then Peter eats spaguetti''.

A program runs sequentially from the first line of code to the last

\end{minipage}
\end{minipage}

\vspace{2mm}\noindent\hrulefill{}

\begin{minipage}{\linewidth}\noindent
{\tiny 016.}\\
\begin{minipage}[t]{.485\linewidth}

What is the output of:

\begin{framed}

\begin{Shaded}
\begin{Highlighting}[]
\DataTypeTok{int}\NormalTok{ var1 = }\DecValTok{6}\NormalTok{;}
\DataTypeTok{int}\NormalTok{ var2 = }\DecValTok{3}\NormalTok{;}
\DataTypeTok{int}\NormalTok{ var3 = }\DecValTok{10}\NormalTok{;}

\NormalTok{var2 = var1*}\DecValTok{3}\NormalTok{;}

\BuiltInTok{std::}\NormalTok{cout}
\NormalTok{  << (var1+var2) * var3 - var}\DecValTok{2}
\NormalTok{  << }\BuiltInTok{std::}\NormalTok{endl;}
\end{Highlighting}
\end{Shaded}

\end{framed}

\end{minipage}
\hfill
\begin{minipage}[t]{.485\linewidth}

The output is:

\begin{framed}

\begin{verbatim}
222
\end{verbatim}

\end{framed}

We can in fact assign to the variable (at the left of \texttt{=}) any
\VERB|\DataTypeTok{int}| value result of any computation. In this case,
the computation \VERB|\NormalTok{var1*}\DecValTok{3}| is assigned to
\VERB|\NormalTok{var}\DecValTok{2}|

\end{minipage}
\end{minipage}

\vspace{2mm}\noindent\hrulefill{}

\begin{minipage}{\linewidth}\noindent
{\tiny 017.}\\
\begin{minipage}[t]{.485\linewidth}

What is the output of:\footnote{Notice the dots (\texttt{...}) in the
  code above. This dots are just for notation, they aren't meant to be
  written in the source code. The dots represent a division between two
  different parts of code.

  \inlinetodo{Add whole file here}}

\begin{framed}

\begin{Shaded}
\begin{Highlighting}[]
\PreprocessorTok{#include }\ImportTok{<cmath>}
\NormalTok{...}
\DataTypeTok{int}\NormalTok{ var1 = }\DecValTok{6}\NormalTok{;}
\DataTypeTok{int}\NormalTok{ var2 = }\DecValTok{3}\NormalTok{;}
\DataTypeTok{int}\NormalTok{ var3 = var1 * pow(var2, }\DecValTok{3}\NormalTok{);}

\BuiltInTok{std::}\NormalTok{cout << }\StringTok{"var1 * pow(var2, 3) => "}
\NormalTok{          << var3 << }\BuiltInTok{std::}\NormalTok{endl;}
\end{Highlighting}
\end{Shaded}

\end{framed}

\end{minipage}
\hfill
\begin{minipage}[t]{.485\linewidth}

The output is:

\begin{framed}

\begin{verbatim}
var1 * pow(var2, 3) => 162
\end{verbatim}

\end{framed}

Remember that

\begin{Shaded}
\begin{Highlighting}[]
\DataTypeTok{int}\NormalTok{ var3 = var1 * pow(var2, }\DecValTok{3}\NormalTok{);}
\end{Highlighting}
\end{Shaded}

is equivalent to

\begin{Shaded}
\begin{Highlighting}[]
\DataTypeTok{int}\NormalTok{ var3;}
\NormalTok{var3 = var1 * pow(var2, }\DecValTok{3}\NormalTok{);}
\end{Highlighting}
\end{Shaded}

\end{minipage}
\end{minipage}

\vspace{2mm}\noindent\hrulefill{}

\end{document}
