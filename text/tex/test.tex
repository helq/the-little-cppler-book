\PassOptionsToPackage{unicode=true}{hyperref} % options for packages loaded elsewhere
\PassOptionsToPackage{hyphens}{url}
%
\documentclass[]{article}
\usepackage{lmodern}
\usepackage{amssymb,amsmath}
\usepackage{ifxetex,ifluatex}
\usepackage{fixltx2e} % provides \textsubscript
\ifnum 0\ifxetex 1\fi\ifluatex 1\fi=0 % if pdftex
  \usepackage[T1]{fontenc}
  \usepackage[utf8]{inputenc}
  \usepackage{textcomp} % provides euro and other symbols
\else % if luatex or xelatex
  \usepackage{unicode-math}
  \defaultfontfeatures{Ligatures=TeX,Scale=MatchLowercase}
\fi
% use upquote if available, for straight quotes in verbatim environments
\IfFileExists{upquote.sty}{\usepackage{upquote}}{}
% use microtype if available
\IfFileExists{microtype.sty}{%
\usepackage[]{microtype}
\UseMicrotypeSet[protrusion]{basicmath} % disable protrusion for tt fonts
}{}
\IfFileExists{parskip.sty}{%
\usepackage{parskip}
}{% else
\setlength{\parindent}{0pt}
\setlength{\parskip}{6pt plus 2pt minus 1pt}
}
\usepackage{hyperref}
\hypersetup{
            pdftitle={The Little CPPler},
            pdfborder={0 0 0},
            breaklinks=true}
\urlstyle{same}  % don't use monospace font for urls
\usepackage[verbose,a4paper,textwidth=190mm,textheight=260mm,voffset=7mm]{geometry}
\usepackage{color}
\usepackage{fancyvrb}
\newcommand{\VerbBar}{|}
\newcommand{\VERB}{\Verb[commandchars=\\\{\}]}
\DefineVerbatimEnvironment{Highlighting}{Verbatim}{commandchars=\\\{\}}
% Add ',fontsize=\small' for more characters per line
\newenvironment{Shaded}{}{}
\newcommand{\AlertTok}[1]{\textcolor[rgb]{1.00,0.00,0.00}{\textbf{#1}}}
\newcommand{\AnnotationTok}[1]{\textcolor[rgb]{0.38,0.63,0.69}{\textbf{\textit{#1}}}}
\newcommand{\AttributeTok}[1]{\textcolor[rgb]{0.49,0.56,0.16}{#1}}
\newcommand{\BaseNTok}[1]{\textcolor[rgb]{0.25,0.63,0.44}{#1}}
\newcommand{\BuiltInTok}[1]{#1}
\newcommand{\CharTok}[1]{\textcolor[rgb]{0.25,0.44,0.63}{#1}}
\newcommand{\CommentTok}[1]{\textcolor[rgb]{0.38,0.63,0.69}{\textit{#1}}}
\newcommand{\CommentVarTok}[1]{\textcolor[rgb]{0.38,0.63,0.69}{\textbf{\textit{#1}}}}
\newcommand{\ConstantTok}[1]{\textcolor[rgb]{0.53,0.00,0.00}{#1}}
\newcommand{\ControlFlowTok}[1]{\textcolor[rgb]{0.00,0.44,0.13}{\textbf{#1}}}
\newcommand{\DataTypeTok}[1]{\textcolor[rgb]{0.56,0.13,0.00}{#1}}
\newcommand{\DecValTok}[1]{\textcolor[rgb]{0.25,0.63,0.44}{#1}}
\newcommand{\DocumentationTok}[1]{\textcolor[rgb]{0.73,0.13,0.13}{\textit{#1}}}
\newcommand{\ErrorTok}[1]{\textcolor[rgb]{1.00,0.00,0.00}{\textbf{#1}}}
\newcommand{\ExtensionTok}[1]{#1}
\newcommand{\FloatTok}[1]{\textcolor[rgb]{0.25,0.63,0.44}{#1}}
\newcommand{\FunctionTok}[1]{\textcolor[rgb]{0.02,0.16,0.49}{#1}}
\newcommand{\ImportTok}[1]{#1}
\newcommand{\InformationTok}[1]{\textcolor[rgb]{0.38,0.63,0.69}{\textbf{\textit{#1}}}}
\newcommand{\KeywordTok}[1]{\textcolor[rgb]{0.00,0.44,0.13}{\textbf{#1}}}
\newcommand{\NormalTok}[1]{#1}
\newcommand{\OperatorTok}[1]{\textcolor[rgb]{0.40,0.40,0.40}{#1}}
\newcommand{\OtherTok}[1]{\textcolor[rgb]{0.00,0.44,0.13}{#1}}
\newcommand{\PreprocessorTok}[1]{\textcolor[rgb]{0.74,0.48,0.00}{#1}}
\newcommand{\RegionMarkerTok}[1]{#1}
\newcommand{\SpecialCharTok}[1]{\textcolor[rgb]{0.25,0.44,0.63}{#1}}
\newcommand{\SpecialStringTok}[1]{\textcolor[rgb]{0.73,0.40,0.53}{#1}}
\newcommand{\StringTok}[1]{\textcolor[rgb]{0.25,0.44,0.63}{#1}}
\newcommand{\VariableTok}[1]{\textcolor[rgb]{0.10,0.09,0.49}{#1}}
\newcommand{\VerbatimStringTok}[1]{\textcolor[rgb]{0.25,0.44,0.63}{#1}}
\newcommand{\WarningTok}[1]{\textcolor[rgb]{0.38,0.63,0.69}{\textbf{\textit{#1}}}}
\setlength{\emergencystretch}{3em}  % prevent overfull lines
\providecommand{\tightlist}{%
  \setlength{\itemsep}{0pt}\setlength{\parskip}{0pt}}
\setcounter{secnumdepth}{0}
% Redefines (sub)paragraphs to behave more like sections
\ifx\paragraph\undefined\else
\let\oldparagraph\paragraph
\renewcommand{\paragraph}[1]{\oldparagraph{#1}\mbox{}}
\fi
\ifx\subparagraph\undefined\else
\let\oldsubparagraph\subparagraph
\renewcommand{\subparagraph}[1]{\oldsubparagraph{#1}\mbox{}}
\fi

% set default figure placement to htbp
\makeatletter
\def\fps@figure{htbp}
\makeatother

\usepackage{framed}
%\renewenvironment{Shaded}{\begin{framed}}{\end{framed}}
%\usepackage{mdframed}
%\renewenvironment{Shaded}{\begin{mdframed}}{\end{mdframed}}

\usepackage{pbox}

%\usepackage{layout} % package to look at the page layout

%\usepackage{fancyhdr}
%\usepackage{lastpage}
%\pagestyle{fancy}
%\lhead{hi}
%\rhead{ho}

\usepackage[l2tabu, orthodox]{nag} % this is to make sure I use some modern LaTeX tactics

%%%
% Custom footnote taken from https://tex.stackexchange.com/a/78227
\makeatletter
\newcommand*{\myfnsymbolsingle}[1]{%
  \ensuremath{%
    \ifcase#1% 0
    \or % 1
      \dagger
    \or % 2
      \ddagger
    \or % 3
      \mathsection
    \else % >= 4
      \@ctrerr
    \fi
  }%
}
\makeatother

\newcommand*{\myfnsymbol}[1]{%
  \myfnsymbolsingle{\value{#1}}%
}

% remove upper boundary by multiplying the symbols if needed
\usepackage{alphalph}
\newalphalph{\myfnsymbolmult}[mult]{\myfnsymbolsingle}{}

\renewcommand*{\thempfootnote}{%
  \myfnsymbolmult{\value{mpfootnote}}%
}

%\renewcommand{\thefootnote}{\fnsymbol{footnote}}
%%%

\usepackage{fmtcount}% for adding padding with zeros
\newcounter{exercise}
\setcounter{exercise}{1}

\title{The Little CPPler}
\date{}

\begin{document}
\maketitle

\hypertarget{basics}{%
\section{Basics}\label{basics}}

\hypertarget{getting-to-know-c11}{%
\subsection{Getting to know C++(11)}\label{getting-to-know-c11}}

\vspace{2mm}\noindent\hrulefill{}

\noindent
{\tiny 001.}\\
\begin{minipage}[t]{.485\linewidth}

What do you think the following code will output after compiling it?

\begin{framed}

\begin{Shaded}
\begin{Highlighting}[]
\BuiltInTok{std::}\NormalTok{cout << }\StringTok{"Here am I!"}\NormalTok{;}
\end{Highlighting}
\end{Shaded}

\end{framed}

\end{minipage}
\hfill
\begin{minipage}[t]{.485\linewidth}

And the output is:

\begin{framed}

\begin{verbatim}
Here am I!
\end{verbatim}

\end{framed}

\end{minipage}

\vspace{2mm}\noindent\hrulefill{}

\noindent
{\tiny 002.}\\
\begin{minipage}[t]{.485\linewidth}

What do you think this would do\footnote{This is a footnote, any time
  there is something for clarification a footnote will be used.}

\begin{framed}

\begin{Shaded}
\begin{Highlighting}[]
\BuiltInTok{std::}\NormalTok{cout << }\StringTok{"It's me papus BR"}\NormalTok{ << }\BuiltInTok{std::}\NormalTok{endl;}
\end{Highlighting}
\end{Shaded}

\end{framed}

\end{minipage}
\hfill
\begin{minipage}[t]{.485\linewidth}

And the output is:

\begin{framed}

\begin{verbatim}
It's me papus BR
\end{verbatim}

\end{framed}

\end{minipage}

\vspace{2mm}\noindent\hrulefill{}

\noindent
{\tiny 003.}\\
\begin{minipage}[t]{.485\linewidth}

End, some text with a footnote

\begin{framed}

\begin{Shaded}
\begin{Highlighting}[]
\BuiltInTok{std::}\NormalTok{cout << }\StringTok{"Another output, yep!"}\NormalTok{ << }\BuiltInTok{std::}\NormalTok{endl;}
\end{Highlighting}
\end{Shaded}

\end{framed}

\end{minipage}
\hfill
\begin{minipage}[t]{.485\linewidth}

Not much here

\begin{framed}

\begin{verbatim}
Another output, yep!
\end{verbatim}

\end{framed}

\end{minipage}

\vspace{2mm}\noindent\hrulefill{}

\noindent
{\tiny 004.}\\
\begin{minipage}[t]{.485\linewidth}

A simple loop:

\begin{framed}

\begin{Shaded}
\begin{Highlighting}[]
\ControlFlowTok{for}\NormalTok{(}\DataTypeTok{int}\NormalTok{ i=}\DecValTok{0}\NormalTok{; i<}\DecValTok{10}\NormalTok{; i++) \{}
  \BuiltInTok{std::}\NormalTok{cout << i << }\BuiltInTok{std::}\NormalTok{endl;}
\NormalTok{\}}
\end{Highlighting}
\end{Shaded}

\end{framed}

\end{minipage}
\hfill
\begin{minipage}[t]{.485\linewidth}

And the output is:

\begin{framed}

\begin{verbatim}
0
1
2
3
4
5
6
7
8
9
\end{verbatim}

\end{framed}

\end{minipage}

\vspace{2mm}\noindent\hrulefill{}

\noindent
{\tiny 005.}\\
\begin{minipage}[t]{.485\linewidth}

vs:

\begin{framed}

\begin{Shaded}
\begin{Highlighting}[]
\ControlFlowTok{for}\NormalTok{(}\DataTypeTok{int}\NormalTok{ i=}\DecValTok{0}\NormalTok{; i<}\DecValTok{10}\NormalTok{; i++) \{}
  \BuiltInTok{std::}\NormalTok{cout << i << }\StringTok{" "}\NormalTok{;}
\NormalTok{\}}
\end{Highlighting}
\end{Shaded}

\end{framed}

\end{minipage}
\hfill
\begin{minipage}[t]{.485\linewidth}

And the output is:

hi

\begin{framed}

\begin{verbatim*}
0 1 2 3 4 5 6 7 8 9 
\end{verbatim*}

\end{framed}

\end{minipage}

\vspace{2mm}\noindent\hrulefill{}

\end{document}
